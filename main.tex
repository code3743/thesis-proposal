\documentclass[onecolumn,12pt]{IEEEtran}

\usepackage{setspace} 
\usepackage{longtable}
\renewcommand{\thetable}{\arabic{table}}

\onehalfspacing  

% ===============================
% Required packages
% ===============================

% Codificación
\usepackage[utf8]{inputenc} % opcional con TeX Live >= 2020
\usepackage[T1]{fontenc}

% Idioma del documento
\usepackage[spanish]{babel}
\selectlanguage{spanish}

% Matemáticas y símbolos
\usepackage{amsmath, amssymb}

% Gráficos y figuras
\usepackage{graphicx}
\graphicspath{{figures/}}

% Tablas y formato
\usepackage{booktabs}
\usepackage{multirow}
\usepackage{array}
\usepackage{tabularx}
\usepackage{float}
\usepackage{siunitx} % opcional para números con decimales
\usepackage{pgfgantt}
\usepackage{xcolor}
\usepackage{pdfpages}


% Citas y enlaces
\usepackage{cite}
\usepackage{url}

% Colores
\usepackage{xcolor}

% Listas personalizadas
\usepackage{enumitem}

% Enlaces clicables
\usepackage[hidelinks]{hyperref}

% Ajustar nombres de figuras y tablas al español
\addto\captionsspanish{%
  \renewcommand{\figurename}{Figura}
  \renewcommand{\tablename}{Tabla}
}

\renewcommand{\arraystretch}{1.6}
% ===============================
% Custom commands
% ===============================

% Nombre del proyecto o universidad (por si lo necesitas repetir)
\newcommand{\university}{Universidad del Valle}
\newcommand{\faculty}{Facultad de Ingeniería}
\newcommand{\projecttitle}{Anteproyecto de Trabajo de Grado}

% Comando para una línea horizontal 
\newcommand{\HRule}{\rule{\linewidth}{0.5mm}}

% Atajos para términos que se repiten
\newcommand{\ie}{i.\,e.}
\newcommand{\eg}{e.\,g.}


\title{Prototipo web para apoyar la enseñanza de Fundamentos de Interpretación y Compilación de Lenguajes de Programación mediante lenguajes orientados por sintaxis}
\author{
  \IEEEauthorblockN{Jota Emilio López Ramírez, Esmeralda Rivas Guzmán}
  \\
  \IEEEauthorblockA{Universidad del Valle\\
  Facultad de Ingeniería\\
  \texttt{jota.lopez@correounivalle.edu.co, esmeralda.rivas@correounivalle.edu.co}}
}

\begin{document}
\maketitle

\renewcommand{\abstractname}{Resumen}

\begin{abstract}
El presente anteproyecto propone el diseño y desarrollo de un prototipo web interactivo orientado a facilitar la comprensión de los fundamentos de interpretación y compilación de lenguajes de programación. El prototipo permitirá a los estudiantes definir gramáticas y visualizar tanto las estructuras sintácticas resultantes (como árboles de sintaxis abstracta) como el ambiente y los cambios en el estado durante el proceso de evaluación conceptual de programas, sin incluir mecanismos de ejecución paso a paso ni funcionalidades de depuración.

El proyecto se enmarca en principios de usabilidad y en enfoques constructivistas del aprendizaje, y busca funcionar como un recurso complementario para la asignatura \textit{Fundamentos de Interpretación y Compilación de Lenguajes de Programación} de la Universidad del Valle, apoyando la comprensión de conceptos clave mediante representaciones gráficas y recursos interactivos.
\end{abstract}

\renewcommand{\IEEEkeywordsname}{Palabras clave}

\begin{IEEEkeywords}
compiladores, intérpretes, herramienta educativa, visualización de procesos 
\end{IEEEkeywords}

\renewcommand{\thesection}{\arabic{section}}
\renewcommand{\thesubsection}{\thesection.\arabic{subsection}}
\renewcommand{\thesubsubsection}{\thesubsection.\arabic{subsubsection}}
\setcounter{secnumdepth}{4}
\renewcommand{\theparagraph}{\thesubsubsection.\arabic{paragraph}}

\section{Introducción}
El presente anteproyecto tiene como propósito establecer las bases teóricas y metodológicas del trabajo de grado propuesto.
Se expone el contexto general, la motivación y la relevancia del problema a abordar.

El proceso de enseñanza del diseño de compiladores ha sido abordado desde múltiples enfoques pedagógicos.
Por ejemplo, Kundra y Sureka \cite{Kundra2016AnER} presentan un enfoque basado en aprendizaje por proyectos y casos para mejorar la comprensión de los conceptos de compiladores.
De manera complementaria, Vegdahl \cite{Vegdahl2000UsingVT} propone el uso de herramientas de visualización para apoyar el aprendizaje de los procesos internos del compilador.

Otros autores como Baldwin \cite{Baldwin2003ACompiler} destacan la utilidad de lenguajes simplificados y compiladores modulares para facilitar la enseñanza práctica del tema, mientras que Mernik y Žumer \cite{Mernik2003AnET} desarrollaron una herramienta educativa específica para la construcción de compiladores.

La importancia de los lenguajes de programación como marco conceptual para el aprendizaje ha sido resaltada desde los primeros trabajos de Feurzeig, Papert y Lawler \cite{Feurzeig01122011}, quienes vinculan la programación con la enseñanza de las matemáticas desde un enfoque constructivista.

Finalmente, investigaciones más recientes como la de Stamenković y Jovanović \cite{Stamenković2024AWE} exploran sistemas web interactivos que permiten la enseñanza de compiladores a través de entornos digitales, integrando elementos de visualización, automatización y simulación educativa.

Estas contribuciones conforman el marco de referencia que sustenta la relevancia y viabilidad del presente proyecto, al evidenciar la evolución de las estrategias pedagógicas aplicadas al ámbito de los compiladores y su enseñanza.

\section{Formulación del Problema}

\subsection{Descripción del Problema}

En el programa de Ingeniería de Sistemas de la Universidad del Valle, la asignatura \textit{Fundamentos de Interpretación y Compilación de Lenguajes de Programación} (FLP) es un componente esencial para el desarrollo de competencias relacionadas con la comprensión de los procesos internos de los lenguajes de programación. Sin embargo, se ha evidenciado que los estudiantes presentan dificultades para asimilar y aplicar los conceptos abordados, los cuales presentan un alto nivel de abstracción y demandan una sólida comprensión teórica.

Entre las principales causas de esta situación se encuentra la falta de herramientas didácticas que faciliten la visualización práctica de los entornos de ejecución, los cambios en variables y los mecanismos de interpretación durante la evaluación de un programa. Aunque la asignatura se apoya en el lenguaje \textit{Racket} y en la librería \textit{EOPL}, que permiten representar estructuras internas como árboles de sintaxis abstracta (AST) y explorar el funcionamiento del intérprete, estas funcionalidades resultan útiles pero su aprovechamiento requiere un dominio técnico previo del entorno y carecen de un componente visual o interactivo que fomente el aprendizaje autónomo.

Diversos estudios han señalado que el uso de herramientas interactivas y representaciones visuales favorece la comprensión de los procesos de compilación e interpretación, al permitir que los estudiantes observen de forma tangible las transformaciones que ocurren en las distintas etapas del procesamiento de un lenguaje \cite{Mernik2003AnET, Vegdahl2000UsingVT, Kundra2016AnER, Stamenković2024AWE}. Estos enfoques, centrados en la visualización y la experimentación práctica, han demostrado ser efectivos para reducir la abstracción inherente a los contenidos y promover un aprendizaje más significativo. Además, se ha evidenciado que, si bien los lenguajes reales y sus intérpretes ofrecen una base sólida para la enseñanza, su orientación hacia la eficiencia y la complejidad técnica dificulta su aprovechamiento con fines pedagógicos, al no exponer claramente los procesos internos que llevan a la ejecución del código \cite{Spigariol2023Aprendiendo}.

\subsection{Definición del Problema}

La ausencia de herramientas didácticas interactivas que complementen los recursos actuales y faciliten la exploración visual de los procesos de interpretación y compilación ha provocado que los estudiantes mantengan dificultades persistentes en la comprensión profunda de los fundamentos de la asignatura. Esta situación se traduce en una dependencia de clases magistrales, una limitada autonomía en el aprendizaje, la adquisición superficial de conocimientos y un bajo desempeño en proyectos finales o en asignaturas posteriores que requieren estos conceptos.

En este contexto, surge la siguiente pregunta de investigación:

\textit{“¿Cómo puede el desarrollo de una herramienta interactiva facilitar la comprensión y aplicación de los contenidos de la asignatura Fundamentos de Interpretación y Compilación de Lenguajes de Programación de la Universidad del Valle, mediante la visualización de procesos internos y la generación de lenguajes orientados por sintaxis?”}

\section{Marco de Referencia}

\subsection{Marco Teórico}
El proyecto se fundamenta en cuatro áreas de conocimiento: teoría de lenguajes formales y compiladores que proporcionan los fundamentos técnicos; teorías del aprendizaje aplicadas a la educación en ciencias de la computación que informan sobre el diseño pedagógico; visualización educativa de procesos computacionales que determinan las estrategias de representación; y principios de usabilidad e interactividad en herramientas educativas que orientan las decisiones de diseño de interfaz.

\subsubsection{Fundamentos de Lenguajes Formales y Teoría de Compiladores}

Los lenguajes de programación se definen formalmente mediante gramáticas libres de contexto (Context-Free Grammars, CFG), clasificadas por Chomsky en el tipo dos de su jerarquía de lenguajes formales \cite{1056813}. Estas gramáticas establecen reglas de producción que determinan cómo los símbolos no terminales pueden reemplazarse por secuencias de símbolos terminales y no terminales, proporcionando una descripción precisa y estructurada de la sintaxis de un lenguaje.

El análisis sintáctico (parsing) verifica si una cadena pertenece al lenguaje definido por la gramática y construye una representación estructural del código fuente. Este proceso genera el árbol de sintaxis abstracta (AST), el cual pasa por alto detalles superficiales y conserva la estructura semántica esencial del programa \cite{10.5555/1177220}. El AST actúa como puente entre la sintaxis y las etapas posteriores del compilador, facilitando el análisis semántico, la optimización y la generación de código.

La interpretación constituye una forma de ejecución en la cual un intérprete recorre el AST evaluando cada nodo según reglas semánticas y utilizando entornos de ejecución (environments) que asocian identificadores con valores \cite{10.5555/1378240}. Finalmente, los lenguajes orientados por sintaxis (syntax-directed languages) establecen una relación formal entre la estructura gramatical y el significado del programa mediante definiciones y esquemas de traducción dirigidos por la sintaxis (Syntax-Directed Definitions y Syntax-Directed Translation Schemes) \cite{10.5555/1177220}. 

En conjunto, estos fundamentos constituyen la base técnica sobre la cual se estructura la enseñanza de los compiladores, permitiendo que los estudiantes comprendan el proceso de traducción y ejecución desde una perspectiva formal y visual.

\subsubsection{Teorías del Aprendizaje Aplicadas a la Educación en Computación}

\subsubsection{Visualización Educativa en Ciencias de la Computación}

\subsubsection{Principios de Diseño de Herramientas Educativas Interactivas}


\subsection{Estado del Arte}
Se analizan las principales investigaciones, proyectos y desarrollos previos que abordan el mismo problema o similares.

\subsection{Antecedentes}
Se describen experiencias y estudios previos relevantes que sirven como punto de partida para este anteproyecto.

\subsection{Marco Conceptual}
En este apartado se definen los términos y conceptos clave utilizados a lo largo del proyecto.

\section{Alcance del Proyecto}

\subsection{Declaración del Alcance}
El proyecto se centrará en el diseño, desarrollo y evaluación de XYZ para mejorar los procesos de ABC.

\subsection{Objetivos}

\subsubsection{Objetivo General}
Desarrollar una propuesta que permita XYZ en el contexto de ABC.

\subsubsection{Objetivos Específicos}
\begin{itemize}
  \item Analizar las características del problema actual.
  \item Diseñar una arquitectura conceptual para XYZ.
  \item Evaluar la viabilidad de la propuesta desarrollada.
\end{itemize}

\subsubsection{Restricciones y Supuestos}
El desarrollo del proyecto se limitará a DEF, bajo el supuesto de que GHI permanecerá constante durante la ejecución.

\section{Metodologías}

\subsection{Metodología de investigación}

La metodología de investigación adoptada en este proyecto se enmarca en un enfoque mixto, combinando revisión sistemática de literatura y desarrollo experimental de un prototipo educativo, con el fin de identificar dificultades conceptuales y validar la utilidad pedagógica de la herramienta. Para guiar el proceso, se utiliza el modelo de Niveles de Madurez Tecnológica (Technology Readiness Levels, TRL), desarrollado por la NASA en la década de 1970 como un sistema estandarizado para evaluar la madurez de tecnologías durante su evolución desde conceptos básicos hasta aplicaciones operativas \cite{salazar-2023}.

A continuación, se presenta una tabla que resume los seis niveles de TRL, adaptados al desarrollo de software educativo para compiladores e intérpretes de lenguajes de programación:

\begin{center}
\footnotesize
\begin{longtable}{|p{1.5cm}|p{6.5cm}|p{6.5cm}|}
\caption[Niveles de madurez tecnológica]{Niveles de madurez tecnológica adaptados al proyecto} \\
\hline
\textbf{TRL} & \textbf{Descripción General} & \textbf{Aplicación en este Proyecto} \\ \hline
\endfirsthead

\hline
\textbf{TRL} & \textbf{Descripción General} & \textbf{Aplicación en este Proyecto} \\ \hline
\endhead

1 & Principios básicos observados y reportados. &
Identificación inicial de conceptos teóricos en gramáticas formales y análisis sintáctico mediante revisión bibliográfica. \\ \hline

2 & Formulación de concepto tecnológico y/o aplicación. &
Análisis de dificultades conceptuales en la asignatura. \\ \hline

3 & Prueba de componentes analíticos y experimentales. &
Caracterización de requisitos funcionales para el editor de gramáticas y generador de AST. \\ \hline

4 & Validación en entorno de laboratorio. &
Integración y pruebas iniciales de componentes en un entorno de desarrollo local. \\ \hline

5 & Validación en entorno relevante. &
Pruebas de prototipo con datos simulados de estudiantes. \\ \hline

6 & Demostración de sistema prototipo en entorno relevante. &
Evaluación con grupo reducido de estudiantes (10–20) en sesiones simuladas de clase, verificando usabilidad y funcionalidad pedagógica. \\ \hline

\end{longtable}
\end{center}


En coherencia con el alcance definido para este trabajo de grado, el proyecto se orienta a alcanzar el TRL 6, correspondiente a un prototipo validado en un entorno relevante. Esto implica que el objetivo no es producir un sistema completamente operativo ni desplegarlo en un entorno institucional de uso permanente, sino implementar y evaluar un prototipo funcional que permita demostrar la viabilidad técnica del enfoque. 

Los niveles superiores se excluyen, dado que requieren validación en escenarios reales de operación, estudios longitudinales y recursos que exceden el tiempo y alcance del proyecto académico.

\subsubsection{Metodología PRISMA para la Revisión Sistemática}

Para la fase inicial de revisión de literatura, se aplicará la guía PRISMA (Preferred Reporting Items for Systematic Reviews and Meta-Analyses) \cite{Prisma2020}. Este enfoque estructurado permitirá identificar, seleccionar y evaluar estudios relevantes sobre herramientas educativas para la enseñanza de compiladores e intérpretes y dificultades conceptuales comunes en estos temas. Los pasos clave incluyen:
\begin{enumerate}
    \item Definición de preguntas de investigación y criterios de inclusión/exclusión.
    \item Búsqueda sistemática en bases de datos académicas (IEEE Xplore, Web of Science, Scopus).
    \item Selección de estudios mediante revisión de títulos, resúmenes y textos completos.
    \item Extracción y síntesis de datos relevantes.
\end{enumerate}

Esta metodología garantizará una base sólida de conocimiento para informar el diseño del prototipo y asegurar que se aborden las necesidades educativas identificadas en la literatura.

\subsection{Metodología de desarrollo de software}

Para el desarrollo del prototipo se adoptará Scrum como marco de trabajo ágil. Aunque Scrum no se considera una metodología rígida de desarrollo de software, su enfoque iterativo basado en ciclos cortos de planificación, ejecución y evaluación lo convierte en una opción adecuada para proyectos que requieren retroalimentación continua y una evolución progresiva del producto \cite{ScrumGuide2020, AtlassianScrum2025}.

La elección de Scrum se sustenta en un análisis previo de metodologías y marcos de trabajo (véase el Anexo~\ref{anexo:analisis}), tomando como referencia criterios de selección establecidos en la literatura, tales como el nivel de formalidad, la flexibilidad, el tamaño del equipo y la naturaleza cambiante de los requisitos \cite{tinoco2010, florez2014}. Con base en estos criterios, Scrum se identificó como una alternativa pertinente para un proyecto académico con un equipo pequeño y un alcance sujeto a ajustes sucesivos.

El proceso de desarrollo se organizará mediante \textit{Sprints} de duración fija, dentro de los cuales se definirán objetivos concretos y se desarrollarán incrementos funcionales del sistema. Scrum proporcionará la estructura para gestionar el avance mediante sus artefactos principales: el \textit{Product Backlog}, que consolida los requisitos; el \textit{Sprint Backlog}, que especifica los compromisos de cada iteración; y el \textit{Incremento}, correspondiente al avance verificable generado al finalizar cada Sprint \cite{ScrumGuide2020}.

Dado el tamaño reducido del equipo, los integrantes asumirán combinadamente las responsabilidades de \textit{Scrum Master} y \textit{Development Team}, mientras que el director del trabajo de grado actuará como \textit{Product Owner}, definiendo prioridades y validando los entregables. Diversos estudios han mostrado que Scrum puede adaptarse efectivamente a equipos pequeños y a entornos académicos o experimentales, lo cual respalda su utilización en este proyecto \cite{Masood2021RealWorldScrum, DiazVargas2018}.

\subsection{Metodología de gestión de actividades}

Para gestionar de forma operativa las actividades del proyecto se adoptará el enfoque Kanban, una metodología visual que permite controlar el trabajo en curso (Work In Progress, WIP), identificar cuellos de botella y maximizar la entrega continua \cite{KanbanScrum2010}. Aunque el desarrollo del software seguirá el marco Scrum, la gestión diaria de tareas se realizará mediante Kanban como herramienta de seguimiento del flujo de trabajo, tanto para las tareas técnicas como de documentación. Esta combinación garantiza trazabilidad, flexibilidad y control del avance del proyecto.

El tablero contendrá como mínimo las siguientes columnas:

\begin{enumerate}
    \item \textbf{Backlog:} tareas identificadas y priorizadas.
    \item \textbf{To Do:} tareas comprometidas en el sprint actual.
    \item \textbf{In Progress:} máximo 3-4 tarjetas simultáneas.
    \item \textbf{Review / Testing:} código o funcionalidad lista para revisión o pruebas.
    \item \textbf{Done:} completada y aceptada.
\end{enumerate}

Con este enfoque se busca mantener un flujo de trabajo eficiente, evitar la sobrecarga y asegurar la calidad en cada etapa del desarrollo del prototipo web educativo.

\subsection{Metodología de evaluación}

La evaluación del prototipo combinará un enfoque formativo con el marco de usabilidad definido por la norma ISO 9241-11. Este marco considera la efectividad en el cumplimiento de objetivos, la eficiencia en términos de esfuerzo y tiempo requeridos, y la satisfacción del usuario, entendida como sus percepciones y actitudes frente al sistema \cite{ISO9241-11}. Este enfoque resulta adecuado para identificar problemas de interacción y para validar la claridad conceptual de las visualizaciones.

La evaluación será de carácter cualitativo predominante y se llevará a cabo en una única ronda al finalizar el ciclo de implementación, con la participación de entre 10 y 20 estudiantes de la asignatura, además del docente responsable, empleando las siguientes herramientas de recolección de datos:

\begin{enumerate}
    \item \textbf{Lista de verificación de funcionalidad}: aplicada por los desarrolladores para verificar el cumplimiento técnico de los requisitos principales.
    \item \textbf{System Usability Scale (SUS)}: cuestionario estandarizado de 10 ítems que proporciona una métrica cuantitativa comparable de usabilidad percibida \cite{brooke1995sus}.
    \item \textbf{Cuestionario de utilidad pedagógica}: preguntas enfocadas en la claridad de las visualizaciones, la utilidad para comprender conceptos de compilación e interpretación, y sugerencias de mejora.
\end{enumerate}

El análisis de los datos combinará el cálculo del puntaje SUS con una síntesis cualitativa de las observaciones recogidas y los comentarios del docente como usuario experto. A partir de esto, los resultados incluirán recomendaciones priorizadas de mejora. Aunque su implementación quede fuera del alcance temporal del proyecto, los resultados constituirán evidencia suficiente para respaldar el logro del TRL 6.

% Sección de cronograma
\section{Presupuesto}

Teniendo en cuenta que el Trabajo de Grado en el marco del programa de Ingeniería de Sistemas, tiene un total de 8 créditos académicos según su contenido curricular y cada crédito según el artículo 11 del Decreto 1295 del Ministerio de Educación Nacional, “equivale a cuarenta y ocho (48) horas de trabajo académico del estudiante, que comprende las horas con acompañamiento directo del docente y las horas de trabajo independiente que el estudiante debe dedicar a la realización de actividades de estudio, prácticas u otras que sean necesarias para alcanzar las metas de aprendizaje” \cite{decreto1295} se establece lo siguiente:

\begin{table}[H]
\centering
\begin{tabular}{|p{4cm}|p{4cm}|c|}
\hline
\multicolumn{2}{|c|}{\textbf{Indicador}} & \textbf{Valor} \\
\hline
\multirow{3}{4cm}{Créditos y horas destinados al trabajo de investigación}
& Créditos académicos & 8 \\
\cline{2-3}
& Horas por crédito & 48 \\
\cline{2-3}
& Total horas de trabajo en investigación & 384 \\
\hline
\end{tabular}
\vspace{2pt}
\caption{Créditos académicos}
\end{table}


\begin{table}[H]
\centering
\begin{tabular}{|p{4cm}|p{5cm}|c|}
\hline
\multicolumn{2}{|c|}{\textbf{Indicador}} & \textbf{Valor} \\
\hline
\multirow{3}{4cm}{Estimación del tiempo para concluir la investigación}
& Número de semanas & 36 \\
\cline{2-3}
& Número de meses (1 mes = 4 semanas) & 9 \\
\cline{2-3}
& Semestres (4 meses = 1 semestre) & 2 \\
\hline
\end{tabular}
\vspace{2pt}
\caption{Estimación de tiempo de la investigación en semanas, meses y semestres}
\end{table}


Así pues, con base en la normatividad sobre la elaboración de proyectos de investigación en la Universidad, se establece que la duración del trabajo de investigación corresponde a 2 semestres. 

Para los cursos de Trabajo de Grado I (TG1) y Trabajo de Grado II (TG2), la asignación de créditos corresponde a 2 y 6 respectivamente. De acuerdo con la normativa institucional, cada crédito equivale a 48 horas de trabajo académico del estudiante. Por tanto, TG1 contempla 96 horas de dedicación del estudiante y TG2 un total de 288 horas. Adicionalmente, según la Resolución 022 de 2001 de la Universidad del Valle, el profesor dispone de 22 horas semestrales para actividades de asesoría en trabajos de investigación de pregrado.

No obstante, el Decreto 1295 del 20 de abril de 2010 establece que, por cada hora de acompañamiento directo, el estudiante debe dedicar al menos dos horas adicionales de trabajo independiente. Con base en este criterio, las horas inicialmente asignadas al estudiante se duplican para reflejar el esfuerzo real requerido en cada curso. La estimación ajustada, que integra tanto las horas del profesor como las horas ampliadas del estudiante, se presenta a continuación:

\begin{table}[H]
\centering
\begin{tabular}{|p{3.5cm}|p{4.5cm}|c|}
\hline
\multicolumn{2}{|c|}{\textbf{Indicador}} & \textbf{Valor (horas)} \\
\hline

\multirow{3}{3.5cm}{TG1}
& Horas del estudiante (ajustadas) & 192 \\
\cline{2-3}
& Horas del profesor & 22 \\
\cline{2-3}
& Total & 214 \\
\hline

\multirow{3}{3.5cm}{TG2}
& Horas del estudiante (ajustadas) & 576 \\
\cline{2-3}
& Horas del profesor & 22 \\
\cline{2-3}
& Total & 598 \\
\hline

\multirow{3}{3.5cm}{Total proyecto}
& Horas del estudiante (ajustadas) & 768 \\
\cline{2-3}
& Horas del profesor & 44 \\
\cline{2-3}
& Total general & 812 \\
\hline

\end{tabular}
\vspace{2pt}
\caption{Horas ajustadas con trabajo adicional para TG1 y TG2}
\end{table}


Con base en la estimación total de horas requeridas para el desarrollo del proyecto, y considerando que este trabajo de investigación es ejecutado por dos estudiantes, se procede a establecer el presupuesto correspondiente. Para ello, se toman como referencia las horas asignadas al profesor asociado y las horas totales de dedicación de cada estudiante, aplicando los valores institucionales vigentes para la remuneración de ambos roles.

\begin{table}[H]
\centering
\begin{tabular}{|p{4cm}|c|c|c|c|}
\hline
\textbf{Rol} & \textbf{Horas del proyecto} & \textbf{Valor por hora} & \textbf{Total} & \textbf{Financiación} \\
\hline

Profesor asociado & 44 & 72\,700 & \$3.198.800 & Universidad \\ \hline

Estudiante & 768 & 6\,189 & \$4.753.152 & Especie \\ \hline

Estudiante & 768 & 6\,189 & \$4.753.152 & Especie \\ \hline

\textbf{TOTAL} & & & \textbf{\$12.705.104} & \\ 
\hline

\end{tabular}
\vspace{2pt}
\caption{Presupuesto total para el desarrollo del proyecto}
\end{table}


A partir del presupuesto de talento humano previamente calculado, se incorpora además el costo del servicio de internet requerido por los dos estudiantes durante los 12 meses de ejecución del proyecto, con un valor mensual de 50.000 por estudiante. Con esta información, se consolida el presupuesto total del proyecto, presentado a continuación:

\begin{table}[H]
\centering
\begin{tabular}{|p{4cm}|c|c|c|}
\hline
\textbf{Concepto} & \textbf{Cantidad} & \textbf{Costo unitario} & \textbf{Total} \\
\hline

Talento humano & - & - & \$13.289.116 \\ \hline

Internet (12 meses, estudiante 1) & 12 & 50\,000 & 600\,000 \\ \hline

Internet (12 meses, estudiante 2) & 12 & 50\,000 & 600\,000 \\ \hline

\textbf{TOTAL GENERAL} & & & \textbf{\$14.489.116} \\ 
\hline

\end{tabular}
\vspace{2pt}
\caption{Presupuesto total del proyecto}
\end{table}

\section{Impacto ambiental de la propuesta}

\bibliographystyle{IEEEtran}
\bibliography{references}


\end{document}
