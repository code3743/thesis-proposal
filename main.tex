\documentclass[onecolumn,12pt]{IEEEtran}

\usepackage{setspace}  
\onehalfspacing  

% ===============================
% Required packages
% ===============================

% Codificación
\usepackage[utf8]{inputenc} % opcional con TeX Live >= 2020
\usepackage[T1]{fontenc}

% Idioma del documento
\usepackage[spanish]{babel}
\selectlanguage{spanish}

% Matemáticas y símbolos
\usepackage{amsmath, amssymb}

% Gráficos y figuras
\usepackage{graphicx}
\graphicspath{{figures/}}

% Tablas y formato
\usepackage{booktabs}
\usepackage{multirow}
\usepackage{array}
\usepackage{tabularx}
\usepackage{float}
\usepackage{siunitx} % opcional para números con decimales
\usepackage{pgfgantt}
\usepackage{xcolor}
\usepackage{pdfpages}


% Citas y enlaces
\usepackage{cite}
\usepackage{url}

% Colores
\usepackage{xcolor}

% Listas personalizadas
\usepackage{enumitem}

% Enlaces clicables
\usepackage[hidelinks]{hyperref}

% Ajustar nombres de figuras y tablas al español
\addto\captionsspanish{%
  \renewcommand{\figurename}{Figura}
  \renewcommand{\tablename}{Tabla}
}

\renewcommand{\arraystretch}{1.6}
% ===============================
% Custom commands
% ===============================

% Nombre del proyecto o universidad (por si lo necesitas repetir)
\newcommand{\university}{Universidad del Valle}
\newcommand{\faculty}{Facultad de Ingeniería}
\newcommand{\projecttitle}{Anteproyecto de Trabajo de Grado}

% Comando para una línea horizontal 
\newcommand{\HRule}{\rule{\linewidth}{0.5mm}}

% Atajos para términos que se repiten
\newcommand{\ie}{i.\,e.}
\newcommand{\eg}{e.\,g.}


\title{Desarrollo de una herramienta interactiva para apoyar la enseñanza de Fundamentos de Interpretación y Compilación de LP mediante lenguajes orientados por sintaxis}
\author{
  \IEEEauthorblockN{Jota Emilio López Ramírez, Esmeralda Rivas Guzmán}
  \IEEEauthorblockA{Universidad del Valle\\
  Facultad de Ingeniería\\
  \texttt{jota.lopez@correounivalle.edu.co, esmeralda.rivas@correounivalle.edu.co}}
}

\begin{document}
\maketitle

\renewcommand{\abstractname}{Resumen}

\begin{abstract}
En este trabajo se presenta el desarrollo de una herramienta interactiva diseñada para apoyar la enseñanza de los Fundamentos de Interpretación y Compilación de Lenguajes de Programación (LP). La herramienta utiliza lenguajes orientados por sintaxis con el fin de facilitar la comprensión de los procesos de análisis y traducción del código fuente. Además, se busca mejorar la experiencia de aprendizaje de los estudiantes mediante la visualización de los componentes del compilador y su funcionamiento interno.
\end{abstract}

\renewcommand{\IEEEkeywordsname}{Palabras clave}

\begin{IEEEkeywords}
Trabajo de grado, Anteproyecto, Investigación, Universidad del Valle, Enseñanza
\end{IEEEkeywords}

\renewcommand{\thesection}{\arabic{section}}
\renewcommand{\thesubsection}{\thesection.\arabic{subsection}}
\renewcommand{\thesubsubsection}{\thesubsection.\arabic{subsubsection}}
\setcounter{secnumdepth}{4}
\renewcommand{\theparagraph}{\thesubsubsection.\arabic{paragraph}}

\section{Introducción}
El presente anteproyecto tiene como propósito establecer las bases teóricas y metodológicas del trabajo de grado propuesto.
Se expone el contexto general, la motivación y la relevancia del problema a abordar.

El proceso de enseñanza del diseño de compiladores ha sido abordado desde múltiples enfoques pedagógicos.
Por ejemplo, Kundra y Sureka \cite{Kundra2016AnER} presentan un enfoque basado en aprendizaje por proyectos y casos para mejorar la comprensión de los conceptos de compiladores.
De manera complementaria, Vegdahl \cite{Vegdahl2000UsingVT} propone el uso de herramientas de visualización para apoyar el aprendizaje de los procesos internos del compilador.

Otros autores como Baldwin \cite{Baldwin2003ACompiler} destacan la utilidad de lenguajes simplificados y compiladores modulares para facilitar la enseñanza práctica del tema, mientras que Mernik y Žumer \cite{Mernik2003AnET} desarrollaron una herramienta educativa específica para la construcción de compiladores.

La importancia de los lenguajes de programación como marco conceptual para el aprendizaje ha sido resaltada desde los primeros trabajos de Feurzeig, Papert y Lawler \cite{Feurzeig01122011}, quienes vinculan la programación con la enseñanza de las matemáticas desde un enfoque constructivista.

Finalmente, investigaciones más recientes como la de Stamenković y Jovanović \cite{Stamenković2024AWE} exploran sistemas web interactivos que permiten la enseñanza de compiladores a través de entornos digitales, integrando elementos de visualización, automatización y simulación educativa.

Estas contribuciones conforman el marco de referencia que sustenta la relevancia y viabilidad del presente proyecto, al evidenciar la evolución de las estrategias pedagógicas aplicadas al ámbito de los compiladores y su enseñanza.

\section{Formulación del Problema}

\subsection{Descripción del Problema}

En el programa de Ingeniería de Sistemas de la Universidad del Valle, la asignatura \textit{Fundamentos de Interpretación y Compilación de Lenguajes de Programación} (FLP) es un componente esencial para el desarrollo de competencias relacionadas con la comprensión de los procesos internos de los lenguajes de programación. Sin embargo, se ha evidenciado que los estudiantes presentan dificultades para asimilar y aplicar los conceptos abordados, los cuales presentan un alto nivel de abstracción y demandan una sólida comprensión teórica.

Entre las principales causas de esta situación se encuentra la falta de herramientas didácticas que faciliten la visualización práctica de los entornos de ejecución, los cambios en variables y los mecanismos de interpretación durante la evaluación de un programa. Aunque la asignatura se apoya en el lenguaje \textit{Racket} y en la librería \textit{EOPL}, que permiten representar estructuras internas como árboles de sintaxis abstracta (AST) y explorar el funcionamiento del intérprete, estas funcionalidades resultan útiles pero su aprovechamiento requiere un dominio técnico previo del entorno y carecen de un componente visual o interactivo que fomente el aprendizaje autónomo.

Diversos estudios han señalado que el uso de herramientas interactivas y representaciones visuales favorece la comprensión de los procesos de compilación e interpretación, al permitir que los estudiantes observen de forma tangible las transformaciones que ocurren en las distintas etapas del procesamiento de un lenguaje \cite{Mernik2003AnET, Vegdahl2000UsingVT, Kundra2016AnER, Stamenković2024AWE}. Estos enfoques, centrados en la visualización y la experimentación práctica, han demostrado ser efectivos para reducir la abstracción inherente a los contenidos y promover un aprendizaje más significativo. Además, se ha evidenciado que, si bien los lenguajes reales y sus intérpretes ofrecen una base sólida para la enseñanza, su orientación hacia la eficiencia y la complejidad técnica dificulta su aprovechamiento con fines pedagógicos, al no exponer claramente los procesos internos que llevan a la ejecución del código \cite{Spigariol2023Aprendiendo}.

\subsection{Definición del Problema}

La ausencia de herramientas didácticas interactivas que complementen los recursos actuales y faciliten la exploración visual de los procesos de interpretación y compilación ha provocado que los estudiantes mantengan dificultades persistentes en la comprensión profunda de los fundamentos de la asignatura. Esta situación se traduce en una dependencia de clases magistrales, una limitada autonomía en el aprendizaje, la adquisición superficial de conocimientos y un bajo desempeño en proyectos finales o en asignaturas posteriores que requieren estos conceptos.

En este contexto, surge la siguiente pregunta de investigación:

\textit{“¿Cómo puede el desarrollo de una herramienta interactiva facilitar la comprensión y aplicación de los contenidos de la asignatura Fundamentos de Interpretación y Compilación de Lenguajes de Programación de la Universidad del Valle, mediante la visualización de procesos internos y la generación de lenguajes orientados por sintaxis?”}

\section{Marco de Referencia}

\subsection{Marco Teórico}
El proyecto se fundamenta en cuatro áreas de conocimiento: teoría de lenguajes formales y compiladores que proporcionan los fundamentos técnicos; teorías del aprendizaje aplicadas a la educación en ciencias de la computación que informan sobre el diseño pedagógico; visualización educativa de procesos computacionales que determinan las estrategias de representación; y principios de usabilidad e interactividad en herramientas educativas que orientan las decisiones de diseño de interfaz.

\subsubsection{Fundamentos de Lenguajes Formales y Teoría de Compiladores}

Los lenguajes de programación se definen formalmente mediante gramáticas libres de contexto (Context-Free Grammars, CFG), clasificadas por Chomsky en el tipo dos de su jerarquía de lenguajes formales \cite{1056813}. Estas gramáticas establecen reglas de producción que determinan cómo los símbolos no terminales pueden reemplazarse por secuencias de símbolos terminales y no terminales, proporcionando una descripción precisa y estructurada de la sintaxis de un lenguaje.

El análisis sintáctico (parsing) verifica si una cadena pertenece al lenguaje definido por la gramática y construye una representación estructural del código fuente. Este proceso genera el árbol de sintaxis abstracta (AST), el cual pasa por alto detalles superficiales y conserva la estructura semántica esencial del programa \cite{10.5555/1177220}. El AST actúa como puente entre la sintaxis y las etapas posteriores del compilador, facilitando el análisis semántico, la optimización y la generación de código.

La interpretación constituye una forma de ejecución en la cual un intérprete recorre el AST evaluando cada nodo según reglas semánticas y utilizando entornos de ejecución (environments) que asocian identificadores con valores \cite{10.5555/1378240}. Finalmente, los lenguajes orientados por sintaxis (syntax-directed languages) establecen una relación formal entre la estructura gramatical y el significado del programa mediante definiciones y esquemas de traducción dirigidos por la sintaxis (Syntax-Directed Definitions y Syntax-Directed Translation Schemes) \cite{10.5555/1177220}. 

En conjunto, estos fundamentos constituyen la base técnica sobre la cual se estructura la enseñanza de los compiladores, permitiendo que los estudiantes comprendan el proceso de traducción y ejecución desde una perspectiva formal y visual.

\subsubsection{Teorías del Aprendizaje Aplicadas a la Educación en Computación}

\subsubsection{Visualización Educativa en Ciencias de la Computación}

\subsubsection{Principios de Diseño de Herramientas Educativas Interactivas}


\subsection{Estado del Arte}
Se analizan las principales investigaciones, proyectos y desarrollos previos que abordan el mismo problema o similares.

\subsection{Antecedentes}
Se describen experiencias y estudios previos relevantes que sirven como punto de partida para este anteproyecto.

\subsection{Marco Conceptual}
En este apartado se definen los términos y conceptos clave utilizados a lo largo del proyecto.

\section{Alcance del Proyecto}

\subsection{Declaración del Alcance}
El proyecto se centrará en el diseño, desarrollo y evaluación de XYZ para mejorar los procesos de ABC.

\subsection{Objetivos}

\subsubsection{Objetivo General}
Desarrollar una propuesta que permita XYZ en el contexto de ABC.

\subsubsection{Objetivos Específicos}
\begin{itemize}
  \item Analizar las características del problema actual.
  \item Diseñar una arquitectura conceptual para XYZ.
  \item Evaluar la viabilidad de la propuesta desarrollada.
\end{itemize}

\subsubsection{Restricciones y Supuestos}
El desarrollo del proyecto se limitará a DEF, bajo el supuesto de que GHI permanecerá constante durante la ejecución.


\bibliographystyle{IEEEtran}
\bibliography{references}


\end{document}
