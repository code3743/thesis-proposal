\section{Metodologías}

\subsection{Metodología de investigación}

La metodología de investigación adoptada en este proyecto se enmarca en un enfoque mixto, combinando revisión sistemática de literatura y desarrollo experimental de un prototipo educativo, con el fin de identificar dificultades conceptuales y validar la utilidad pedagógica de la herramienta. Para guiar el proceso, se utiliza el modelo de Niveles de Madurez Tecnológica (Technology Readiness Levels, TRL), desarrollado por la NASA en la década de 1970 como un sistema estandarizado para evaluar la madurez de tecnologías durante su evolución desde conceptos básicos hasta aplicaciones operativas \cite{salazar-2023}.

A continuación, se presenta una tabla que resume los seis niveles de TRL, adaptados al desarrollo de software educativo para compiladores e intérpretes de lenguajes de programación:

\begin{longtable}{|p{1.0cm}|p{7.5cm}|p{7.9cm}|}
\hline
\textbf{TRL} & \textbf{Descripción General} & \textbf{Aplicación en este Proyecto} \\ \hline
\endfirsthead

\hline
\textbf{TRL} & \textbf{Descripción General} & \textbf{Aplicación en este Proyecto} \\ \hline
\endhead

1 & Principios básicos observados y reportados. & Identificación inicial de conceptos teóricos en gramáticas formales y análisis sintáctico mediante revisión bibliográfica. \\ \hline
2 & Formulación de concepto tecnológico y/o aplicación. & Análisis de dificultades conceptuales en la asignatura. \\ \hline
3 & Prueba de componentes analíticos y experimentales. & Caracterización de requisitos funcionales para el editor de gramáticas y generador de AST. \\ \hline
4 & Validación en entorno de laboratorio. & Integración y pruebas iniciales de componentes en un entorno de desarrollo local. \\ \hline
5 & Validación en entorno relevante. & Pruebas de prototipo con datos simulados de estudiantes. \\ \hline
6 & Demostración de sistema prototipo en entorno relevante. & Evaluación con grupo reducido de estudiantes (10–20) en sesiones simuladas de clase, verificando usabilidad y funcionalidad pedagógica. \\ \hline

\caption{Niveles de Madurez Tecnológica adaptados al proyecto.}
\label{tab:trl}
\end{longtable}
\vspace{-0.8em}
\noindent

En coherencia con el alcance definido para este trabajo de grado, el proyecto se orienta a alcanzar el TRL 6, correspondiente a un prototipo validado en un entorno relevante. Esto implica que el objetivo no es producir un sistema completamente operativo ni desplegarlo en un entorno institucional de uso permanente, sino implementar y evaluar un prototipo funcional que permita demostrar la viabilidad técnica del enfoque. 

Los niveles superiores se excluyen, dado que requieren validación en escenarios reales de operación, estudios longitudinales y recursos que exceden el tiempo y alcance del proyecto académico.

\subsubsection{Metodología PRISMA para la Revisión Sistemática}

Para la fase inicial de revisión de literatura, se aplicará la guía PRISMA (Preferred Reporting Items for Systematic Reviews and Meta-Analyses) \cite{Prisma2020}. Este enfoque estructurado permitirá identificar, seleccionar y evaluar estudios relevantes sobre herramientas educativas para la enseñanza de compiladores e intérpretes y dificultades conceptuales comunes en estos temas. Los pasos clave incluyen:
\begin{enumerate}
    \item Definición de preguntas de investigación y criterios de inclusión/exclusión.
    \item Búsqueda sistemática en bases de datos académicas (IEEE Xplore, Web of Science, Scopus).
    \item Selección de estudios mediante revisión de títulos, resúmenes y textos completos.
    \item Extracción y síntesis de datos relevantes.
\end{enumerate}

Esta metodología garantizará una base sólida de conocimiento para informar el diseño del prototipo y asegurar que se aborden las necesidades educativas identificadas en la literatura.

\subsection{Metodología de desarrollo de software}

\subsection{Metodología de gestión de actividades}

Para gestionar de forma operativa las actividades del proyecto se adoptará el enfoque Kanban, una metodología visual que permite controlar el trabajo en curso (Work In Progress, WIP), identificar cuellos de botella y maximizar la entrega continua \cite{KanbanScrum2010}. Aunque el desarrollo del software seguirá el marco Scrum, la gestión diaria de tareas se realizará mediante Kanban como herramienta de seguimiento del flujo de trabajo, tanto para las tareas técnicas como de documentación. Esta combinación garantiza trazabilidad, flexibilidad y control del avance del proyecto.

El tablero contendrá como mínimo las siguientes columnas:

\begin{enumerate}
    \item \textbf{Backlog:} tareas identificadas y priorizadas.
    \item \textbf{To Do:} tareas comprometidas en el sprint actual.
    \item \textbf{In Progress:} máximo 3-4 tarjetas simultáneas.
    \item \textbf{Review / Testing:} código o funcionalidad lista para revisión o pruebas.
    \item \textbf{Done:} completada y aceptada.
\end{enumerate}

Con este enfoque se busca mantener un flujo de trabajo eficiente, evitar la sobrecarga y asegurar la calidad en cada etapa del desarrollo del prototipo web educativo.

\subsection{Metodología de evaluación}

La evaluación del prototipo combinará un enfoque formativo con el marco de usabilidad definido por la norma ISO 9241-11. Este marco considera la efectividad en el cumplimiento de objetivos, la eficiencia en términos de esfuerzo y tiempo requeridos, y la satisfacción del usuario, entendida como sus percepciones y actitudes frente al sistema \cite{ISO9241-11}. Este enfoque resulta adecuado para identificar problemas de interacción y para validar la claridad conceptual de las visualizaciones.

La evaluación será de carácter cualitativo predominante y se llevará a cabo en una única ronda al finalizar el ciclo de implementación, con la participación de entre 10 y 20 estudiantes de la asignatura, además del docente responsable, empleando las siguientes herramientas de recolección de datos:

\begin{enumerate}
    \item \textbf{Lista de verificación de funcionalidad}: aplicada por los desarrolladores para verificar el cumplimiento técnico de los requisitos principales.
    \item \textbf{System Usability Scale (SUS)}: cuestionario estandarizado de 10 ítems que proporciona una métrica cuantitativa comparable de usabilidad percibida \cite{brooke1995sus}.
    \item \textbf{Cuestionario de utilidad pedagógica}: preguntas enfocadas en la claridad de las visualizaciones, la utilidad para comprender conceptos de compilación e interpretación, y sugerencias de mejora.
\end{enumerate}

El análisis de los datos combinará el cálculo del puntaje SUS con una síntesis cualitativa de las observaciones recogidas y los comentarios del docente como usuario experto. A partir de esto, los resultados incluirán recomendaciones priorizadas de mejora. Aunque su implementación quede fuera del alcance temporal del proyecto, los resultados constituirán evidencia suficiente para respaldar el logro del TRL 6.
