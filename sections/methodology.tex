\section{Metodologías}

\subsection{Metodología de investigación}

La metodología de investigación adoptada en este proyecto se enmarca en un enfoque mixto, combinando revisión sistemática de literatura y desarrollo experimental de un prototipo educativo, con el fin de identificar dificultades conceptuales y validar la utilidad pedagógica de la herramienta. Para guiar el proceso, se utiliza el modelo de Niveles de Madurez Tecnológica (Technology Readiness Levels, TRL), desarrollado por la NASA en la década de 1970 como un sistema estandarizado para evaluar la madurez de tecnologías durante su evolución desde conceptos básicos hasta aplicaciones operativas \cite{salazar-2023}.

A continuación, se presenta una tabla que resume los seis niveles de TRL, adaptados al desarrollo de software educativo para compiladores e intérpretes de lenguajes de programación:

\begin{longtable}{|p{1.0cm}|p{7.5cm}|p{7.9cm}|}
\hline
\textbf{TRL} & \textbf{Descripción General} & \textbf{Aplicación en este Proyecto} \\ \hline
\endfirsthead

\hline
\textbf{TRL} & \textbf{Descripción General} & \textbf{Aplicación en este Proyecto} \\ \hline
\endhead

1 & Principios básicos observados y reportados. & Identificación inicial de conceptos teóricos en gramáticas formales y análisis sintáctico mediante revisión bibliográfica. \\ \hline
2 & Formulación de concepto tecnológico y/o aplicación. & Análisis de dificultades conceptuales en la asignatura. \\ \hline
3 & Prueba de componentes analíticos y experimentales. & Caracterización de requisitos funcionales para el editor de gramáticas y generador de AST. \\ \hline
4 & Validación en entorno de laboratorio. & Integración y pruebas iniciales de componentes en un entorno de desarrollo local. \\ \hline
5 & Validación en entorno relevante. & Pruebas de prototipo con datos simulados de estudiantes. \\ \hline
6 & Demostración de sistema prototipo en entorno relevante. & Evaluación con grupo reducido de estudiantes (10–20) en sesiones simuladas de clase, verificando usabilidad y funcionalidad pedagógica. \\ \hline

\caption{Niveles de Madurez Tecnológica adaptados al proyecto.}
\label{tab:trl}
\end{longtable}
\vspace{-0.8em}
\noindent

En coherencia con el alcance definido para este trabajo de grado, el proyecto se orienta a alcanzar el TRL 6, correspondiente a un prototipo validado en un entorno relevante. Esto implica que el objetivo no es producir un sistema completamente operativo ni desplegarlo en un entorno institucional de uso permanente, sino implementar y evaluar un prototipo funcional que permita demostrar la viabilidad técnica del enfoque. 

Los niveles superiores se excluyen, dado que requieren validación en escenarios reales de operación, estudios longitudinales y recursos que exceden el tiempo y alcance del proyecto académico.







