\section{Presupuesto}

Teniendo en cuenta que el Trabajo de Grado en el marco del programa de Ingeniería de Sistemas, tiene un total de 8 créditos académicos según su contenido curricular y cada crédito según el artículo 11 del Decreto 1295 del Ministerio de Educación Nacional, “equivale a cuarenta y ocho (48) horas de trabajo académico del estudiante, que comprende las horas con acompañamiento directo del docente y las horas de trabajo independiente que el estudiante debe dedicar a la realización de actividades de estudio, prácticas u otras que sean necesarias para alcanzar las metas de aprendizaje” \cite{decreto1295} se establece lo siguiente:

\begin{table}[H]
\centering
\begin{tabular}{|p{4cm}|p{4cm}|c|}
\hline
\multicolumn{2}{|c|}{\textbf{Indicador}} & \textbf{Valor} \\
\hline
\multirow{3}{4cm}{Créditos y horas destinados al trabajo de investigación}
& Créditos académicos & 8 \\
\cline{2-3}
& Horas por crédito & 48 \\
\cline{2-3}
& Total horas de trabajo en investigación & 384 \\
\hline
\end{tabular}
\vspace{2pt}
\caption{Créditos académicos}
\end{table}


\begin{table}[H]
\centering
\begin{tabular}{|p{4cm}|p{5cm}|c|}
\hline
\multicolumn{2}{|c|}{\textbf{Indicador}} & \textbf{Valor} \\
\hline
\multirow{3}{4cm}{Estimación del tiempo para concluir la investigación}
& Número de semanas & 36 \\
\cline{2-3}
& Número de meses (1 mes = 4 semanas) & 9 \\
\cline{2-3}
& Semestres (4 meses = 1 semestre) & 2 \\
\hline
\end{tabular}
\vspace{2pt}
\caption{Estimación de tiempo de la investigación en semanas, meses y semestres}
\end{table}


Así pues, con base en la normatividad sobre la elaboración de proyectos de investigación en la Universidad, se establece que la duración del trabajo de investigación corresponde a 2 semestres. 

Para los cursos de Trabajo de Grado I (TG1) y Trabajo de Grado II (TG2), la asignación de créditos corresponde a 2 y 6 respectivamente. De acuerdo con la normativa institucional, cada crédito equivale a 48 horas de trabajo académico del estudiante. Por tanto, TG1 contempla 96 horas de dedicación del estudiante y TG2 un total de 288 horas. Adicionalmente, según la Resolución 022 de 2001 de la Universidad del Valle, el profesor dispone de 22 horas semestrales para actividades de asesoría en trabajos de investigación de pregrado.

No obstante, el Decreto 1295 del 20 de abril de 2010 establece que, por cada hora de acompañamiento directo, el estudiante debe dedicar al menos dos horas adicionales de trabajo independiente. Con base en este criterio, las horas inicialmente asignadas al estudiante se duplican para reflejar el esfuerzo real requerido en cada curso. La estimación ajustada, que integra tanto las horas del profesor como las horas ampliadas del estudiante, se presenta a continuación:

\begin{table}[H]
\centering
\begin{tabular}{|p{3.5cm}|p{4.5cm}|c|}
\hline
\multicolumn{2}{|c|}{\textbf{Indicador}} & \textbf{Valor (horas)} \\
\hline

\multirow{3}{3.5cm}{TG1}
& Horas del estudiante (ajustadas) & 192 \\
\cline{2-3}
& Horas del profesor & 22 \\
\cline{2-3}
& Total & 214 \\
\hline

\multirow{3}{3.5cm}{TG2}
& Horas del estudiante (ajustadas) & 576 \\
\cline{2-3}
& Horas del profesor & 22 \\
\cline{2-3}
& Total & 598 \\
\hline

\multirow{3}{3.5cm}{Total proyecto}
& Horas del estudiante (ajustadas) & 768 \\
\cline{2-3}
& Horas del profesor & 44 \\
\cline{2-3}
& Total general & 812 \\
\hline

\end{tabular}
\vspace{2pt}
\caption{Horas ajustadas con trabajo adicional para TG1 y TG2}
\end{table}


Con base en la estimación total de horas requeridas para el desarrollo del proyecto, y considerando que este trabajo de investigación es ejecutado por dos estudiantes, se procede a establecer el presupuesto correspondiente. Para ello, se toman como referencia las horas asignadas al profesor asociado y las horas totales de dedicación de cada estudiante, aplicando los valores institucionales vigentes para la remuneración de ambos roles.

\begin{table}[H]
\centering
\begin{tabular}{|p{4cm}|c|c|c|c|}
\hline
\textbf{Rol} & \textbf{Horas del proyecto} & \textbf{Valor por hora} & \textbf{Total} & \textbf{Financiación} \\
\hline

Profesor asociado & 44 & 72\,700 & \$3.198.800 & Universidad \\ \hline

Estudiante & 768 & 6\,189 & \$4.753.152 & Especie \\ \hline

Estudiante & 768 & 6\,189 & \$4.753.152 & Especie \\ \hline

\textbf{TOTAL} & & & \textbf{\$12.705.104} & \\ 
\hline

\end{tabular}
\vspace{2pt}
\caption{Presupuesto total para el desarrollo del proyecto}
\end{table}


A partir del presupuesto de talento humano previamente calculado, se incorpora además el costo del servicio de internet requerido por los dos estudiantes durante los 12 meses de ejecución del proyecto, con un valor mensual de 50.000 por estudiante. Con esta información, se consolida el presupuesto total del proyecto, presentado a continuación:

\begin{table}[H]
\centering
\begin{tabular}{|p{4cm}|c|c|c|}
\hline
\textbf{Concepto} & \textbf{Cantidad} & \textbf{Costo unitario} & \textbf{Total} \\
\hline

Talento humano & - & - & \$13.289.116 \\ \hline

Internet (12 meses, estudiante 1) & 12 & 50\,000 & 600\,000 \\ \hline

Internet (12 meses, estudiante 2) & 12 & 50\,000 & 600\,000 \\ \hline

\textbf{TOTAL GENERAL} & & & \textbf{\$14.489.116} \\ 
\hline

\end{tabular}
\vspace{2pt}
\caption{Presupuesto total del proyecto}
\end{table}
