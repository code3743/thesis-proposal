\section{Alcance del proyecto}

Este proyecto comprende el diseño y desarrollo de un \textit{prototipo web educativo} orientado a apoyar la comprensión de conceptos fundamentales de interpretación y compilación de lenguajes de programación. El alcance técnico incluye: (1) la definición y edición de \textit{gramáticas con sintaxis configurable} por el usuario, (2) la visualización de \textit{árboles de sintaxis abstracta (AST)} generados a partir de dichas gramáticas, y (3) la representación de los \textit{cambios en el ambiente} durante la evaluación de programas escritos bajo la gramática definida.

\subsection{Declaración del alcance}

El prototipo se limita a funcionalidades demostrativas orientadas al aprendizaje, sin la intención de soportar un lenguaje completo de propósito general ni mecanismos avanzados de ejecución.

La investigación se desarrolla en el contexto de la asignatura \textit{Fundamentos de Interpretación y Compilación de Lenguajes de Programación} del programa de Ingeniería de Sistemas de la Universidad del Valle, sede Tuluá, con una población de estudio compuesta por los estudiantes matriculados en la asignatura y el docente responsable, no obstante, debido al tiempo disponible para la realización del trabajo de grado, la evaluación del prototipo se llevará a cabo con un grupo reducido de entre 10 y 20 estudiantes, quienes participarán en pruebas de funcionalidad, usabilidad y retroalimentación cualitativa, lo que exige planear, ejecutar y analizar las pruebas en un período acotado.

El proyecto adopta un enfoque \textit{experimental y evaluativo}, que comprende el análisis documental, el diseño e implementación del prototipo web y la realización de pruebas cualitativas centradas en la comprensión conceptual y la experiencia de uso. Por otro lado, quedan fuera del alcance aspectos como la depuración paso a paso, la compilación a código máquina, la autenticación o persistencia de usuarios, el despliegue en infraestructura productiva, la optimización de rendimiento y las evaluaciones pedagógicas de impacto a largo plazo.


\subsubsection{Entregables del Proyecto}
\begin{enumerate}
    \item \textbf{Prototipo web funcional:} Despliegue en un entorno de pruebas, con las capacidades definidas en el alcance.

    \item \textbf{Documentación técnica:} Descripción de la arquitectura, tecnologías utilizadas, decisiones de diseño y guía de instalación/despliegue.

    \item \textbf{Manual de usuario:} Guía de uso dirigida a estudiantes y docentes, que explique cómo utilizar las funcionalidades del prototipo.

    \item \textbf{Código fuente:} Repositorio con el código completo, debidamente documentado y versionado.
\end{enumerate}


\subsection{Objetivos}

\subsubsection{Objetivo General}
Desarrollar un prototipo web que apoye la comprensión de los contenidos de la asignatura \textit{Fundamentos de Interpretación y Compilación de Lenguajes de Programación (LP)} de la Universidad del Valle mediante la visualización de estructuras internas y el uso de lenguajes orientados por sintaxis.
\subsubsection{Objetivos Específicos}
\begin{enumerate}
\item Identificar las dificultades conceptuales que enfrentan los estudiantes de la asignatura, con el propósito de orientar la selección de ejemplos, casos de uso y mecanismos de visualización.
\item Caracterizar los fundamentos teóricos y técnicos relacionados con las gramáticas formales, el análisis sintáctico y la generación de árboles de sintaxis abstracta, que servirán como base conceptual para el diseño del prototipo.
\item Diseñar la arquitectura web y las interfaces de usuario, integrando estrategias de visualización que favorezcan la comprensión de las estructuras sintácticas.
\item Implementar un prototipo funcional que permita definir gramáticas, visualizar los árboles de sintaxis abstracta y los cambios en el estado durante la evaluación de los programas escritos por el usuario.
\item Evaluar la usabilidad y funcionalidad del prototipo mediante pruebas con estudiantes inscritos en la asignatura y con el docente responsable.
\end{enumerate}

\subsubsection{Restricciones y Supuestos}

\paragraph{Restricciones temporales}

El proyecto debe completarse en un período total de ocho meses, lo cual limita el alcance de las funcionalidades a implementar y la profundidad de la evaluación.

\paragraph{Restricciones de recursos humanos}

El desarrollo estará a cargo de dos estudiantes de trabajo de grado, lo que restringe la complejidad y cantidad de características que pueden incorporarse. La participación de estudiantes en las pruebas de usabilidad depende del calendario académico y de su disponibilidad voluntaria durante el período de evaluación.

\paragraph{Restricciones tecnológicas}

El prototipo debe funcionar en navegadores web modernos sin requerir instalaciones adicionales, garantizando compatibilidad con los dispositivos comúnmente utilizados por los estudiantes, asimismo, no se cuenta con infraestructura institucional garantizada para el despliegue, por lo que este se limitará a servicios gratuitos o de bajo costo. La selección tecnológica debe considerar la curva de aprendizaje del equipo y privilegiar herramientas con documentación amplia y comunidades activas.

\paragraph{Restricciones de evaluación}

No es posible medir el impacto en el aprendizaje a largo plazo debido a la necesidad de estudios experimentales prolongados, con grupos de control y seguimiento por varios semestres. Por lo tanto, la evaluación se restringirá a funcionalidad técnica y usabilidad, mediante cuestionarios o herramientas similares, y se prevé una muestra entre 10 y 20 estudiantes, para identificar problemas de interacción pero no para análisis estadísticos sobre efectividad pedagógica.

\paragraph{Supuestos del contexto educativo}

Se asume que los estudiantes cuentan con conocimientos básicos de programación y estructuras de datos. También se supone la disposición de un grupo de estudiantes para participar voluntariamente en las pruebas, así como la colaboración del docente para facilitar acceso a la asignatura y permitir la presentación del prototipo.

\paragraph{Supuestos técnicos}

Se asume la viabilidad técnica de implementar un parser para gramáticas libres de contexto y un generador de AST con rendimiento adecuado para uso interactivo. Se presupone que las tecnologías web modernas permitirán desarrollar visualizaciones dinámicas sin dependencias externas ni instalación adicional en el navegador.

\paragraph{Supuestos sobre recursos y disponibilidad}

Se asume que será posible acceder a estudiantes de un período académico durante la fase de evaluación. También se presupone que el docente proporcionará material didáctico relevante para orientar el diseño de la herramienta. Asimismo, se considera que existirán servicios de alojamiento gratuitos o de bajo costo suficientes para desplegar el prototipo durante la etapa de pruebas.

\paragraph{Supuestos sobre el problema y la evaluación}

Se asume que las dificultades conceptuales identificadas en la literatura y en la consulta con el docente reflejan los obstáculos reales que enfrentan los estudiantes. También se presupone que la visualización de estructuras sintácticas (AST) aporta un valor pedagógico potencial, aun cuando este no se mida formalmente. Finalmente, se asume que los criterios de funcionalidad técnica y de usabilidad constituyen indicadores válidos y suficientes para evaluar el prototipo dentro del marco temporal del proyecto, aun cuando no sustituyen estudios de impacto en el aprendizaje.

\subsubsection{Resultados Esperados}

\paragraph{Resultados del Objetivo Específico 1}

Se espera obtener una caracterización documentada de las dificultades conceptuales que enfrentan los estudiantes, mediante revisión de literatura, cuestionarios y análisis de material académico disponible, con el propósito de orientar la construcción de ejemplos pedagógicos, logrando identificar al menos cinco conceptos que servirán como base para la biblioteca de casos predefinidos que se incluirá en el prototipo.

\paragraph{Resultados del Objetivo Específico 2}

Se definirá y documentará un conjunto priorizado de entre 8 y 12 requisitos funcionales del sistema y entre 5 y 8 requisitos no funcionales acompañados de criterios de aceptación específicos que permitan verificar su cumplimiento.

\paragraph{Resultados del Objetivo Específico 3}

Se determinará la arquitectura del prototipo especificando los componentes principales, sus responsabilidades y el flujo de datos a través del sistema. Además, se producirán prototipos de interfaz de usuario y se definirán las estrategias de visualización seleccionadas, incluyendo bocetos o diagramas ilustrativos.

Igualmente, se realizará la selección de tecnologías a utilizar, justificando cada elección en términos de la curva de aprendizaje del equipo, disponibilidad de documentación, viabilidad de implementar las funcionalidades requeridas, y compatibilidad entre tecnologías seleccionadas.

\paragraph{Resultados del Objetivo Específico 4}

Se obtendrá un prototipo funcional desplegado en un entorno de pruebas dando cumplimiento a los requisitos funcionales definidos en el \textit{objetivo específico 2}.

El código fuente estará alojado en un repositorio GitHub con historial de commits, debidamente documentado y acompañado de archivo README con instrucciones de instalación y ejecución.

\paragraph{Resultados del Objetivo Específico 5}

Se realizará una verificación básica del funcionamiento técnico y una valoración preliminar de su usabilidad, con un conjunto limitado de pruebas que permitan confirmar que el prototipo se ejecuta correctamente sus funciones principales.

Además, se obtendrá una percepción inicial de uso por parte del docente y de un grupo reducido de estudiantes, reflejada en un conjunto de observaciones sobre la facilidad para completar tareas básicas y los aspectos que resulten útiles o que requieran mejoras para desarrollos futuros.