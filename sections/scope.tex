\section{Alcance del Proyecto}

\subsection{Declaración del Alcance}
El proyecto se centrará en el diseño, desarrollo y evaluación de XYZ para mejorar los procesos de ABC.

\subsection{Objetivos}

\subsubsection{Objetivo General}
Desarrollar una herramienta interactiva que facilite la comprensión y aplicación de los contenidos de la asignatura \textit{Fundamentos de Interpretación y Compilación de Lenguajes de Programación (LP)} de la Universidad del Valle, mediante la visualización de procesos internos y la generación de lenguajes orientados por sintaxis.

\subsubsection{Objetivos Específicos}
\begin{itemize}
  \item Identificar las dificultades conceptuales y metodológicas que enfrentan los estudiantes de Ingeniería de Sistemas en la asignatura \textit{Fundamentos de Interpretación y Compilación de LP}.
  \item Caracterizar los elementos teóricos y técnicos necesarios para la construcción de intérpretes orientados por sintaxis que favorezcan su aplicación en contextos pedagógicos.
  \item Diseñar una estrategia arquitectónica y didáctica para la construcción de intérpretes orientados por sintaxis, integrando representaciones visuales y mecanismos de interacción que promuevan el aprendizaje autónomo.
  \item Implementar una herramienta interactiva que permita definir gramáticas, visualizar la estructura sintáctica de programas y el seguimiento de su evaluación en los entornos de ejecución.
  \item Evaluar la funcionalidad técnica y la usabilidad de la herramienta desarrollada mediante pruebas con estudiantes de la asignatura \textit{Fundamentos de Interpretación y Compilación de LP}.
\end{itemize}

\subsubsection{Restricciones y Supuestos}

Las siguientes restricciones y supuestos delimitan el alcance y las condiciones del proyecto.

\paragraph{Restricciones temporales}
\begin{itemize}
    \item El proyecto debe completarse en el tiempo estipulado para un trabajo de grado, lo que limita el alcance de las funcionalidades y el tiempo para la validación del sistema.
    \item La evaluación con estudiantes debe realizarse dentro de un período académico específico coordinado con el calendario de la asignatura.
\end{itemize}

\paragraph{Restricciones de recursos humanos}
\begin{itemize}
    \item El desarrollo será realizado por un equipo de dos estudiantes de trabajo de grado, lo cual limita la complejidad y cantidad de características que pueden implementarse.
    \item La disponibilidad de estudiantes para las pruebas dependerá del ciclo académico y de su participación voluntaria.
\end{itemize}

\paragraph{Restricciones tecnológicas}
\begin{itemize}
    \item La herramienta debe ser accesible sin requerir instalación de software especializado por parte de los usuarios finales, priorizando plataformas web o de fácil acceso.
    \item El sistema deberá funcionar en equipos con recursos computacionales estándar utilizados por los estudiantes universitarios y el docente.
    \item No se dispone de infraestructura de servidores institucionales garantizada, por lo que el despliegue se limitará a servicios gratuitos o de bajo costo.
\end{itemize}

\paragraph{Restricciones de alcance funcional}
\begin{itemize}
    \item El sistema se enfocará en la interpretación de lenguajes orientados por sintaxis, sin incluir procesos de compilación avanzada ni optimización de código.
    \item El sistema soportará un subconjunto representativo de características de lenguajes de programación, sin pretender ser un entorno de desarrollo completo.
    \item La visualización se centrará en los conceptos fundamentales de la asignatura, sin abarcar todos los temas del curso.
\end{itemize}

\paragraph{Restricciones de evaluación}
\begin{itemize}
    \item No se podrá medir el impacto en el aprendizaje a largo plazo, debido a que ello requeriría estudios longitudinales fuera del alcance temporal del proyecto.
    \item La evaluación se centrará en aspectos de funcionalidad técnica, usabilidad y percepción de utilidad por parte de los estudiantes y el docente.
\end{itemize}

\paragraph{Supuestos del contexto educativo}
\begin{itemize}
    \item Los estudiantes cuentan con conocimientos básicos en programación y lenguajes de programación.
    \item Existe interés y disposición por parte de los estudiantes en el uso de herramientas complementarias para el aprendizaje.
    \item El docente de la asignatura está dispuesto a colaborar en la evaluación de la herramienta y facilitar su uso con estudiantes.
\end{itemize}

\paragraph{Supuestos técnicos}
\begin{itemize}
    \item Las tecnologías seleccionadas permitirán desarrollar visualizaciones dinámicas y comprensibles para los usuarios.
\end{itemize}

\paragraph{Supuestos de recursos y disponibilidad}
\begin{itemize}
    \item Se contará con acceso a los estudiantes y al material didáctico de la asignatura durante al menos un período académico.
    \item Existirán recursos mínimos de infraestructura para desplegar la herramienta durante la evaluación.
\end{itemize}

\paragraph{Supuestos sobre el problema y la evaluación}
\begin{itemize}
    \item La visualización de procesos internos aporta valor pedagógico y refuerza la comprensión teórica.
    \item Los criterios de funcionalidad, usabilidad y percepción de utilidad son indicadores válidos para evaluar la calidad del prototipo.
\end{itemize}

