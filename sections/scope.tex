\section{Alcance del Proyecto}

\subsection{Declaración del Alcance}
El proyecto se centrará en el diseño, desarrollo y evaluación de XYZ para mejorar los procesos de ABC.

\subsection{Objetivos}

\subsubsection{Objetivo General}
Desarrollar una herramienta interactiva que facilite la comprensión y aplicación de los contenidos de la asignatura \textit{Fundamentos de Interpretación y Compilación de Lenguajes de Programación (LP)} de la Universidad del Valle, mediante la visualización de procesos internos y la generación de lenguajes orientados por sintaxis.

\subsubsection{Objetivos Específicos}
\begin{itemize}
  \item Identificar las dificultades conceptuales y metodológicas que enfrentan los estudiantes de Ingeniería de Sistemas en la asignatura \textit{Fundamentos de Interpretación y Compilación de LP}.
  \item Caracterizar los elementos teóricos y técnicos necesarios para la construcción de intérpretes orientados por sintaxis que favorezcan su aplicación en contextos pedagógicos.
  \item Diseñar una estrategia arquitectónica y didáctica para la construcción de intérpretes orientados por sintaxis, integrando representaciones visuales y mecanismos de interacción que promuevan el aprendizaje autónomo.
  \item Implementar una herramienta interactiva que permita definir gramáticas, visualizar la estructura sintáctica de programas y el seguimiento de su evaluación en los entornos de ejecución.
  \item Evaluar la funcionalidad técnica y la usabilidad de la herramienta desarrollada mediante pruebas con estudiantes de la asignatura \textit{Fundamentos de Interpretación y Compilación de LP}.
\end{itemize}

\subsubsection{Restricciones y Supuestos}
El desarrollo del proyecto se limitará a DEF, bajo el supuesto de que GHI permanecerá constante durante la ejecución.
