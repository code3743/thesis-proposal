\section{Alcance del Proyecto}

\subsection{Declaración del Alcance}
El presente proyecto tiene como alcance el diseño, desarrollo y evaluación de una herramienta interactiva orientada a facilitar la comprensión de conceptos fundamentales de interpretación y compilación de lenguajes de programación, específicamente como apoyo al proceso de enseñanza-aprendizaje en la asignatura \textit{Fundamentos de Interpretación y Compilación de LP} de la Universidad del Valle.

\subsubsection{Alcance del Producto}

\begin{itemize}
    \item Definición y análisis de gramáticas que describan la sintaxis de lenguajes orientados por sintaxis.
    \item Representación visual de estructuras sintácticas derivadas de programas escritos por el usuario.
    \item Ejecución controlada e interactiva de programas, mostrando el proceso de evaluación paso a paso.
    \item Interfaz accesible y didáctica que promueva la exploración y el aprendizaje autónomo.
    \item Inclusión de ejemplos predefinidos que ilustren conceptos centrales de la asignatura.
    \item Posibilidad de guardar y reutilizar gramáticas o programas definidos dentro del entorno, mediante almacenamiento local o mecanismos equivalentes.
\end{itemize}

\subsubsection{Alcance del Proyecto}

\begin{itemize}
    \item \textbf{Análisis:} identificación de las principales dificultades que enfrentan los estudiantes, revisión de herramientas similares y definición de requisitos funcionales y no funcionales.
    \item \textbf{Diseño:} elaboración de la arquitectura del sistema, interfaces de usuario y de las estrategias de visualización de procesos.
    \item \textbf{Implementación:} desarrollo del prototipo funcional de la herramienta con base en los requisitos definidos.
    \item \textbf{Evaluación:} validación técnica y de usabilidad mediante pruebas con estudiantes y el docente de la asignatura.
\end{itemize}

\subsubsection{Fuera del Alcance}

\begin{itemize}
    \item Compilación a código ejecutable o generación de binarios.
    \item Optimización de programas o análisis de rendimiento.
    \item Estudios experimentales que midan impacto en el aprendizaje a largo plazo.
    \item Sistema de gestión de usuarios, autenticación o colaboración multiusuario.
    \item Infraestructura de despliegue de tipo productivo o con alta disponibilidad.
\end{itemize}

\subsubsection{Criterios de Aceptación}

\begin{itemize}
    \item La herramienta permite definir y analizar gramáticas de manera funcional y comprensible.
    \item Se generan visualizaciones coherentes de estructuras sintácticas y del proceso de evaluación.
    \item Los estudiantes y el docente reconocen su utilidad como recurso complementario de aprendizaje.
    \item Se entrega documentación suficiente que describa el diseño, uso y resultados del proyecto.
\end{itemize}

\subsubsection{Entregables del Proyecto}

\begin{enumerate}
    \item \textbf{Herramienta interactiva funcional:} Sistema desplegado y funcional que implementa las capacidades especificadas en el alcance del producto.
    \item \textbf{Casos de ejemplo:} Conjunto de gramáticas y programas de ejemplo que ilustren conceptos de la asignatura.
    \item \textbf{Documentación técnica:} Descripción de la arquitectura, tecnologías utilizadas, decisiones de diseño y guía de instalación/despliegue.
    \item \textbf{Manual de usuario:} Guía de uso de la herramienta orientada a estudiantes y docentes.
    \item \textbf{Informe de evaluación:} Documento con resultados de las pruebas técnicas y de usabilidad, incluyendo retroalimentación de usuarios.
    \item \textbf{Código fuente:} Repositorio con el código completo del sistema, debidamente documentado y versionado.
\end{enumerate}

\subsection{Objetivos}

El planteamiento de los objetivos de este proyecto se apoya en la elaboración de un árbol de objetivos (Figura~\ref{fig:arbol_objetivos}), el cual permite representar la relación jerárquica entre los fines y medios que orientan la propuesta. 

En el centro del esquema se encuentra el propósito general: el desarrollo de una herramienta interactiva que facilite la comprensión y aplicación de los contenidos de la asignatura \textit{Fundamentos de Interpretación y Compilación de Lenguajes de Programación}. 

A partir de este fin se derivan tres ejes de acción: 
\begin{itemize}
    \item La visualización de procesos internos de interpretación y compilación, que se vincula con el diseño e implementación de mecanismos que representen gráficamente la estructura sintáctica de los programas y la evolución de los entornos de ejecución.  
    \item El apoyo visual y didáctico para la comprensión de conceptos abstractos, mediante estrategias que promuevan la exploración y comparación de los procesos de interpretación y compilación.  
    \item El fortalecimiento del aprendizaje autónomo, favorecido por el uso de laboratorios interactivos, ejemplos progresivos y recursos digitales que estimulen la autoevaluación.  
\end{itemize}

Estos ejes contribuyen a metas como la autonomía y motivación de los estudiantes, el fortalecimiento de la comprensión aplicada de los contenidos y el mejor desempeño en cursos posteriores que dependen de estos fundamentos. En conjunto, el árbol de objetivos orienta la estructura de diseño y desarrollo del proyecto, articulando los objetivos específicos con las metas pedagógicas del programa académico.

\begin{figure}[htbp]
    \centering
    \includegraphics[width=0.95\columnwidth]{figures/objective_tree.drawio.png}
    \caption{Árbol de objetivos del proyecto.}
    \label{fig:arbol_objetivos}
\end{figure}

\subsubsection{Objetivo General}
Desarrollar una herramienta interactiva que facilite la comprensión y aplicación de los contenidos de la asignatura \textit{Fundamentos de Interpretación y Compilación de Lenguajes de Programación (LP)} de la Universidad del Valle, mediante la visualización de procesos internos y la generación de lenguajes orientados por sintaxis.

\subsubsection{Objetivos Específicos}
\begin{itemize}
  \item Identificar las dificultades conceptuales y metodológicas que enfrentan los estudiantes de Ingeniería de Sistemas en la asignatura \textit{Fundamentos de Interpretación y Compilación de LP}.
  \item Caracterizar los elementos teóricos y técnicos necesarios para la construcción de intérpretes orientados por sintaxis que favorezcan su aplicación en contextos pedagógicos.
  \item Diseñar una estrategia arquitectónica y didáctica para la construcción de intérpretes orientados por sintaxis, integrando representaciones visuales y mecanismos de interacción que promuevan el aprendizaje autónomo.
  \item Implementar una herramienta interactiva que permita definir gramáticas, visualizar la estructura sintáctica de programas y el seguimiento de su evaluación en los entornos de ejecución.
  \item Evaluar la funcionalidad técnica y la usabilidad de la herramienta desarrollada mediante pruebas con estudiantes de la asignatura \textit{Fundamentos de Interpretación y Compilación de LP}.
\end{itemize}

\subsubsection{Restricciones y Supuestos}

\paragraph{Restricciones temporales}
\begin{itemize}
    \item El proyecto debe completarse en el tiempo estipulado para un trabajo de grado, lo que limita el alcance de las funcionalidades y el tiempo para la validación del sistema.
    \item La evaluación con estudiantes debe realizarse dentro de un período académico específico coordinado con el calendario de la asignatura.
\end{itemize}

\paragraph{Restricciones de recursos humanos}
\begin{itemize}
    \item El desarrollo será realizado por un equipo de dos estudiantes de trabajo de grado, lo cual limita la complejidad y cantidad de características que pueden implementarse.
    \item La disponibilidad de estudiantes para las pruebas dependerá del ciclo académico y de su participación voluntaria.
\end{itemize}

\paragraph{Restricciones tecnológicas}
\begin{itemize}
    \item La herramienta debe ser accesible sin requerir instalación de software especializado por parte de los usuarios finales, priorizando plataformas web o de fácil acceso.
    \item El sistema deberá funcionar en equipos con recursos computacionales estándar utilizados por los estudiantes universitarios y el docente.
    \item No se dispone de infraestructura de servidores institucionales garantizada, por lo que el despliegue se limitará a servicios gratuitos o de bajo costo.
\end{itemize}

\paragraph{Restricciones de alcance funcional}
\begin{itemize}
    \item El sistema se enfocará en la interpretación de lenguajes orientados por sintaxis, sin incluir procesos de compilación avanzada ni optimización de código.
    \item El sistema soportará un subconjunto representativo de características de lenguajes de programación, sin pretender ser un entorno de desarrollo completo.
    \item La visualización se centrará en los conceptos fundamentales de la asignatura, sin abarcar todos los temas del curso.
\end{itemize}

\paragraph{Restricciones de evaluación}
\begin{itemize}
    \item No se podrá medir el impacto en el aprendizaje a largo plazo, debido a que ello requeriría estudios longitudinales fuera del alcance temporal del proyecto.
    \item La evaluación se centrará en aspectos de funcionalidad técnica, usabilidad y percepción de utilidad por parte de los estudiantes y el docente.
\end{itemize}

\paragraph{Supuestos del contexto educativo}
\begin{itemize}
    \item Los estudiantes cuentan con conocimientos básicos en programación y lenguajes de programación.
    \item Existe interés y disposición por parte de los estudiantes en el uso de herramientas complementarias para el aprendizaje.
    \item El docente de la asignatura está dispuesto a colaborar en la evaluación de la herramienta y facilitar su uso con estudiantes.
\end{itemize}

\paragraph{Supuestos técnicos}
\begin{itemize}
    \item Las tecnologías seleccionadas permitirán desarrollar visualizaciones dinámicas y comprensibles para los usuarios.
\end{itemize}

\paragraph{Supuestos de recursos y disponibilidad}
\begin{itemize}
    \item Se contará con acceso a los estudiantes y al material didáctico de la asignatura durante al menos un período académico.
    \item Existirán recursos mínimos de infraestructura para desplegar la herramienta durante la evaluación.
\end{itemize}

\paragraph{Supuestos sobre el problema y la evaluación}
\begin{itemize}
    \item La visualización de procesos internos aporta valor pedagógico y refuerza la comprensión teórica.
    \item Los criterios de funcionalidad, usabilidad y percepción de utilidad son indicadores válidos para evaluar la calidad del prototipo.
\end{itemize}

