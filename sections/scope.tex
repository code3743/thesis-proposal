\section{Alcance del Proyecto}

\subsection{Declaración del Alcance}
El presente proyecto tiene como alcance el diseño, desarrollo y evaluación de usabilidad de una herramienta interactiva orientada a facilitar la comprensión de conceptos fundamentales de interpretación y compilación de lenguajes de programación, específicamente como apoyo al proceso de enseñanza-aprendizaje dentro de la asignatura \textit{Fundamentos de Interpretación y Compilación de LP} de la Universidad del Valle. 

\subsubsection{Alcance del Producto}
La herramienta permitirá la definición y análisis de gramáticas que describan la estructura formal de lenguajes orientados por sintaxis. Asimismo, integrará la representación visual de las estructuras sintácticas (AST) derivadas de los programas escritos por el usuario, con el fin de facilitar la comprensión de su organización interna.

La interfaz será accesible y promoverá la exploración autónoma mediante ejemplos predefinidos que ilustren conceptos centrales de la asignatura. Finalmente, se habilitará el almacenamiento local de gramáticas y programas definidos por el usuario, permitiendo su consulta y reutilización en sesiones posteriores.

\subsubsection{Alcance del Proyecto}
El proyecto se estructura en cuatro fases principales que se ejecutarán dentro del período de ocho meses establecido. La fase de análisis comprende la revisión de literatura relevante, la consulta con el docente de la asignatura y la identificación de los requisitos funcionales y no funcionales del sistema. Esta fase incluye también el reconocimiento de las dificultades más comunes que enfrentan los estudiantes, con el fin de orientar la selección y construcción de ejemplos que acompañarán la herramienta, teniendo en cuenta que las funcionalidades principales como la visualización del AST y el manejo de lenguajes orientados por sintaxis no dependen de fuentes externas adicionales.

La fase de diseño abarca la elaboración de la arquitectura del sistema, el diseño de interfaces de usuario mediante prototipos y la definición de estrategias de visualización adecuadas para explicar conceptos abstractos. Esta fase considerará explícitamente la curva de aprendizaje asociada con las tecnologías seleccionadas, dedicando tiempo suficiente para que el equipo desarrolle competencia técnica antes de proceder a la implementación.

La fase de implementación consiste en el desarrollo del prototipo funcional con base en los requisitos y diseños establecidos, integrando el análisis de gramáticas, la generación de árboles sintácticos abstractos (AST), la interfaz de usuario y los mecanismos de visualización interactiva.

La fase de evaluación comprende la validación de usabilidad mediante pruebas con estudiantes y el docente de la asignatura. Dada la restricción temporal y la necesidad de coordinar con el calendario académico, la evaluación se realizará durante un período académico específico. Se hará a través de cuestionarios de usabilidad y observación de uso con una muestra estimada entre 10 y 20 estudiantes. Esto es suficiente para identificar problemas de interacción y oportunidades de mejora sin pretender medir el impacto en el aprendizaje formal.

\subsubsection{Fuera del Alcance}

\begin{itemize}
    \item Compilación a código ejecutable o generación de binarios.
    \item Optimización de programas o análisis de rendimiento.
    \item Estudios experimentales que midan impacto en el aprendizaje a largo plazo.
    \item Sistema de gestión de usuarios, autenticación o colaboración multiusuario.
    \item Visualización del proceso de evaluación de programas paso a paso.
    \item Infraestructura de despliegue de tipo productivo o con alta disponibilidad.
\end{itemize}

\subsubsection{Criterios de Aceptación}

\begin{itemize}
    \item La herramienta permite definir y analizar gramáticas de manera funcional y comprensible.
    \item Se generan visualizaciones coherentes de estructuras sintácticas.
    \item Los estudiantes y el docente reconocen su utilidad como recurso complementario de aprendizaje.
    \item Se entrega documentación suficiente que describa el diseño, uso y resultados del proyecto.
\end{itemize}

\subsubsection{Entregables del Proyecto}

\begin{enumerate}
    \item \textbf{Herramienta interactiva funcional:} Sistema desplegado en un entorno de pruebas que implementa las capacidades especificadas en el alcance del producto.
    \item \textbf{Documentación técnica:} Descripción de la arquitectura, tecnologías utilizadas, decisiones de diseño y guía de instalación/despliegue.
    \item \textbf{Manual de usuario:} Guía de uso de la herramienta orientada a estudiantes y docentes.
    \item \textbf{Informe de evaluación:} Documento con resultados de las pruebas de usabilidad, incluyendo retroalimentación de usuarios.
    \item \textbf{Código fuente:} Repositorio con el código completo del sistema, debidamente documentado y versionado.
\end{enumerate}

\subsection{Objetivos}

\subsubsection{Objetivo General}
Desarrollar una herramienta interactiva que facilite la comprensión y aplicación de los contenidos de la asignatura \textit{Fundamentos de Interpretación y Compilación de Lenguajes de Programación (LP)} de la Universidad del Valle, mediante la visualización de procesos internos y la generación de lenguajes orientados por sintaxis.

\subsubsection{Objetivos Específicos}
\begin{enumerate}
  \item Identificar las dificultades conceptuales y metodológicas que enfrentan los estudiantes en la asignatura.
  \item Caracterizar los elementos teóricos y técnicos necesarios para la construcción de intérpretes orientados por sintaxis que favorezcan su aplicación en contextos pedagógicos.
  \item Diseñar una estrategia arquitectónica para la construcción de intérpretes orientados por sintaxis, integrando representaciones visuales y mecanismos de interacción que promuevan el aprendizaje autónomo.
  \item Implementar una herramienta interactiva que permita definir gramáticas, visualizar la estructura sintáctica de programas y el seguimiento de su evaluación en sus procesos internos.
  \item Evaluar la funcionalidad técnica y la usabilidad de la herramienta desarrollada mediante pruebas con estudiantes y el docente de la asignatura.
\end{enumerate}

\subsubsection{Restricciones y Supuestos}

\paragraph{Restricciones temporales}
\begin{itemize}
    \item El proyecto debe completarse en el tiempo estipulado para un trabajo de grado, lo que limita el alcance de las funcionalidades y el tiempo para la validación del sistema.
    \item La evaluación con estudiantes debe realizarse dentro de un período académico específico coordinado con el calendario de la asignatura.
\end{itemize}

\paragraph{Restricciones de recursos humanos}
\begin{itemize}
    \item El desarrollo será realizado por un equipo de dos estudiantes de trabajo de grado, lo cual limita la complejidad y cantidad de características que pueden implementarse.
    \item La disponibilidad de estudiantes para las pruebas dependerá del ciclo académico y de su participación voluntaria.
\end{itemize}

\paragraph{Restricciones tecnológicas}
\begin{itemize}
    \item La herramienta debe ser accesible sin requerir instalación de software especializado por parte de los usuarios finales, priorizando plataformas web o de fácil acceso.
    \item El sistema deberá funcionar en equipos con recursos computacionales estándar utilizados por los estudiantes universitarios y el docente.
    \item No se dispone de infraestructura de servidores institucionales garantizada, por lo que el despliegue se limitará a servicios gratuitos o de bajo costo.
\end{itemize}

\paragraph{Restricciones de alcance funcional}
\begin{itemize}
    \item El sistema se enfocará en la interpretación de lenguajes orientados por sintaxis, sin incluir procesos de compilación avanzada ni optimización de código.
    \item El sistema soportará un subconjunto representativo de características de lenguajes de programación, sin pretender ser un entorno de desarrollo completo.
    \item La visualización se centrará en los conceptos fundamentales de la asignatura, sin abarcar todos los temas del curso.
\end{itemize}

\paragraph{Restricciones de evaluación}
\begin{itemize}
    \item No se podrá medir el impacto en el aprendizaje a largo plazo, debido a que ello requeriría estudios longitudinales fuera del alcance temporal del proyecto.
    \item La evaluación se centrará en aspectos de funcionalidad técnica, usabilidad y percepción de utilidad por parte de los estudiantes y el docente.
\end{itemize}

\paragraph{Supuestos del contexto educativo}
\begin{itemize}
    \item Los estudiantes cuentan con conocimientos básicos en programación y lenguajes de programación.
    \item Existe interés y disposición por parte de los estudiantes en el uso de herramientas complementarias para el aprendizaje.
    \item El docente de la asignatura está dispuesto a colaborar en la evaluación de la herramienta y facilitar su uso con estudiantes.
\end{itemize}

\paragraph{Supuestos técnicos}
\begin{itemize}
    \item Las tecnologías seleccionadas permitirán desarrollar visualizaciones dinámicas y comprensibles para los usuarios.
\end{itemize}

\paragraph{Supuestos de recursos y disponibilidad}
\begin{itemize}
    \item Se contará con acceso a los estudiantes y al material didáctico de la asignatura durante al menos un período académico.
    \item Existirán recursos mínimos de infraestructura para desplegar la herramienta durante la evaluación.
\end{itemize}

\paragraph{Supuestos sobre el problema y la evaluación}
\begin{itemize}
    \item La visualización de procesos internos aporta valor pedagógico y refuerza la comprensión teórica.
    \item Los criterios de funcionalidad, usabilidad y percepción de utilidad son indicadores válidos para evaluar la calidad del prototipo.
\end{itemize}

