\section{Descripción del Problema}

En el programa de Ingeniería de Sistemas de la Universidad del Valle, la asignatura \textit{Fundamentos de Interpretación y Compilación de Lenguajes de Programación} (FLP) es un componente esencial para el desarrollo de competencias relacionadas con la comprensión de los procesos internos de los lenguajes de programación. Sin embargo, se ha evidenciado que los estudiantes presentan dificultades para asimilar y aplicar los conceptos abordados, los cuales presentan un alto nivel de abstracción y demandan una sólida comprensión teórica.

Entre las principales causas de esta situación se encuentra la falta de herramientas didácticas que faciliten la visualización práctica de los entornos de ejecución, los cambios en variables y los mecanismos de interpretación durante la evaluación de un programa. Aunque la asignatura se apoya en el lenguaje \textit{Racket} y en la librería \textit{EOPL}, que permiten representar estructuras internas como árboles de sintaxis abstracta (AST) y explorar el funcionamiento del intérprete, estas funcionalidades resultan útiles pero su aprovechamiento requiere un dominio técnico previo del entorno y carecen de un componente visual o interactivo que fomente el aprendizaje autónomo.

Diversos estudios han señalado que el uso de herramientas interactivas y representaciones visuales favorece la comprensión de los procesos de compilación e interpretación, al permitir que los estudiantes observen de forma tangible las transformaciones que ocurren en las distintas etapas del procesamiento de un lenguaje \cite{Mernik2003AnET, Vegdahl2000UsingVT, Kundra2016AnER, Stamenković2024AWE}. Estos enfoques, centrados en la visualización y la experimentación práctica, han demostrado ser efectivos para reducir la abstracción inherente a los contenidos y promover un aprendizaje más significativo. Además, se ha evidenciado que, si bien los lenguajes reales y sus intérpretes ofrecen una base sólida para la enseñanza, su orientación hacia la eficiencia y la complejidad técnica dificulta su aprovechamiento con fines pedagógicos, al no exponer claramente los procesos internos que llevan a la ejecución del código \cite{Spigariol2023Aprendiendo}.

Adicionalmente, con el fin de sustentar estas observaciones desde la experiencia directa de los estudiantes, se realizó una encuesta en la que participaron 27 alumnos que cursaron la asignatura en diferentes periodos académicos. Los resultados evidenciaron que, aunque algunos afirmaron haber comprendido los conceptos teóricos, una proporción considerable expresó incertidumbre o dificultades al aplicarlos y al enfrentarse a los contenidos más abstractos del curso. También se identificó que Racket y la librería EOPL no fueron percibidos como herramientas para apoyar la comprensión, y que la mayoría manifestó dificultades para encontrar recursos externos complementarios. De manera especialmente significativa, la totalidad de los participantes coincidió en que recursos adicionales habrían mejorado de forma notable su proceso de aprendizaje, tal como se detalla en el Anexo~\ref{annex:survey}.,

\section{Definición del Problema}

La ausencia de herramientas didácticas interactivas que complementen los recursos actuales y faciliten la exploración visual de los procesos de interpretación y compilación ha provocado que los estudiantes mantengan dificultades persistentes en la comprensión profunda de los fundamentos de la asignatura. Esta situación se traduce en una dependencia de clases magistrales, una limitada autonomía en el aprendizaje, la adquisición superficial de conocimientos y un bajo desempeño en proyectos finales o en asignaturas posteriores que requieren estos conceptos.

En este contexto, surge la siguiente pregunta de investigación:

\textit{“¿Cómo desarrollar un prototipo web que apoye la comprensión de los contenidos de la asignatura \textit{Fundamentos de Interpretación y Compilación de Lenguajes de Programación (LP)} de la Universidad del Valle mediante la visualización de estructuras internas y el uso de lenguajes orientados por sintaxis?”}
