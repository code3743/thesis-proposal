\section{Introducción}
El presente anteproyecto tiene como propósito establecer las bases teóricas y metodológicas del trabajo de grado propuesto.
Se expone el contexto general, la motivación y la relevancia del problema a abordar.

El proceso de enseñanza del diseño de compiladores ha sido abordado desde múltiples enfoques pedagógicos.
Por ejemplo, Kundra y Sureka \cite{Kundra2016AnER} presentan un enfoque basado en aprendizaje por proyectos y casos para mejorar la comprensión de los conceptos de compiladores.
De manera complementaria, Vegdahl \cite{Vegdahl2000UsingVT} propone el uso de herramientas de visualización para apoyar el aprendizaje de los procesos internos del compilador.

Otros autores como Baldwin \cite{Baldwin2003ACompiler} destacan la utilidad de lenguajes simplificados y compiladores modulares para facilitar la enseñanza práctica del tema, mientras que Mernik y Žumer \cite{Mernik2003AnET} desarrollaron una herramienta educativa específica para la construcción de compiladores.

La importancia de los lenguajes de programación como marco conceptual para el aprendizaje ha sido resaltada desde los primeros trabajos de Feurzeig, Papert y Lawler \cite{Feurzeig01122011}, quienes vinculan la programación con la enseñanza de las matemáticas desde un enfoque constructivista.

Finalmente, investigaciones más recientes como la de Stamenković y Jovanović \cite{Stamenković2024AWE} exploran sistemas web interactivos que permiten la enseñanza de compiladores a través de entornos digitales, integrando elementos de visualización, automatización y simulación educativa.

Estas contribuciones conforman el marco de referencia que sustenta la relevancia y viabilidad del presente proyecto, al evidenciar la evolución de las estrategias pedagógicas aplicadas al ámbito de los compiladores y su enseñanza.
