El campo de la Interpretación y Compilación de Lenguajes de Programación constituye una pieza fundamental en la formación de ingenieros de sistemas; sin embargo, sus contenidos suelen ser percibidos como abstractos y complejos por los estudiantes.

La principal dificultad radica en el alto nivel de abstracción de conceptos esenciales, tales como el análisis sintáctico, la gestión de Árboles de Sintaxis Abstracta (AST) y la representación del ambiente de ejecución de programas. Esta complejidad ha motivado la búsqueda de enfoques pedagógicos alternativos. Por ejemplo, la literatura especializada ha resaltado la efectividad del aprendizaje por proyectos para mejorar la comprensión de conceptos \cite{Kundra2016AnER}, así como la utilidad de lenguajes simplificados y compiladores modulares para facilitar la enseñanza práctica del tema \cite{Baldwin2003ACompiler}.

En este contexto, la visualización de los procesos internos del compilador ha emergido como una estrategia pedagógica fundamental \cite{Vegdahl2000UsingVT}. Si bien existen herramientas educativas \cite{Mernik2003AnET}, en el entorno de la Universidad del Valle aún se evidencia una limitada disponibilidad de recursos interactivos que permitan a los estudiantes manipular y observar de forma práctica y dinámica la evolución del ambiente conceptual de los programas.

La relevancia del proyecto radica en su potencial para complementar las estrategias de enseñanza tradicionales, promoviendo la participación activa del estudiante y fortaleciendo la comprensión de los contenidos de la asignatura \textit{Fundamentos de Interpretación y Compilación de Lenguajes de Programación}. Investigaciones recientes confirman la viabilidad y el impacto positivo de la implementación de sistemas web interactivos en entornos digitales \cite{Stamenković2024AWE}, lo que permite sentar las bases para futuros proyectos de innovación educativa en el área de programación y refuerza la formación práctica de los estudiantes en lenguajes de programación.