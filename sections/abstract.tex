El presente anteproyecto propone el diseño y desarrollo de un prototipo web interactivo orientado a facilitar la comprensión de los fundamentos de interpretación y compilación de lenguajes de programación. El prototipo permitirá a los estudiantes definir gramáticas y visualizar tanto las estructuras sintácticas resultantes (como árboles de sintaxis abstracta) como el ambiente y los cambios en el estado durante el proceso de evaluación conceptual de programas, sin incluir mecanismos de ejecución paso a paso ni funcionalidades de depuración.

El proyecto se enmarca en principios de usabilidad y en enfoques constructivistas del aprendizaje, y busca funcionar como un recurso complementario para la asignatura \textit{Fundamentos de Interpretación y Compilación de Lenguajes de Programación} de la Universidad del Valle, apoyando la comprensión de conceptos clave mediante representaciones gráficas y recursos interactivos.