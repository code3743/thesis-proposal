\section{Marco de Referencia}

\subsection{Marco Teórico}
El proyecto se fundamenta en cuatro áreas de conocimiento: teoría de lenguajes formales y compiladores que proporcionan los fundamentos técnicos; teorías del aprendizaje aplicadas a la educación en ciencias de la computación que informan sobre el diseño pedagógico; visualización educativa de procesos computacionales que determinan las estrategias de representación; y principios de usabilidad e interactividad en herramientas educativas que orientan las decisiones de diseño de interfaz.

\subsubsection{Fundamentos de Lenguajes Formales y Teoría de Compiladores}

Los lenguajes de programación se definen formalmente mediante gramáticas libres de contexto (Context-Free Grammars, CFG), clasificadas por Chomsky en el tipo dos de su jerarquía de lenguajes formales \cite{1056813}. Estas gramáticas establecen reglas de producción que determinan cómo los símbolos no terminales pueden reemplazarse por secuencias de símbolos terminales y no terminales, proporcionando una descripción precisa y estructurada de la sintaxis de un lenguaje.

El análisis sintáctico (parsing) verifica si una cadena pertenece al lenguaje definido por la gramática y construye una representación estructural del código fuente. Este proceso genera el árbol de sintaxis abstracta (AST), el cual pasa por alto detalles superficiales y conserva la estructura semántica esencial del programa \cite{10.5555/1177220}. El AST actúa como puente entre la sintaxis y las etapas posteriores del compilador, facilitando el análisis semántico, la optimización y la generación de código.

La interpretación constituye una forma de ejecución en la cual un intérprete recorre el AST evaluando cada nodo según reglas semánticas y utilizando entornos de ejecución (environments) que asocian identificadores con valores \cite{10.5555/1378240}. Finalmente, los lenguajes orientados por sintaxis (syntax-directed languages) establecen una relación formal entre la estructura gramatical y el significado del programa mediante definiciones y esquemas de traducción dirigidos por la sintaxis (Syntax-Directed Definitions y Syntax-Directed Translation Schemes) \cite{10.5555/1177220}. 

En conjunto, estos fundamentos constituyen la base técnica sobre la cual se estructura la enseñanza de los compiladores, permitiendo que los estudiantes comprendan el proceso de traducción y ejecución desde una perspectiva formal y visual.

\subsubsection{Teorías del Aprendizaje Aplicadas a la Educación en Computación}

\subsubsection{Visualización Educativa en Ciencias de la Computación}

\subsubsection{Principios de Diseño de Herramientas Educativas Interactivas}


\subsection{Estado del Arte}
Se analizan las principales investigaciones, proyectos y desarrollos previos que abordan el mismo problema o similares.

\subsection{Antecedentes}
Se describen experiencias y estudios previos relevantes que sirven como punto de partida para este anteproyecto.

\subsection{Marco Conceptual}
En este apartado se definen los términos y conceptos clave utilizados a lo largo del proyecto.
