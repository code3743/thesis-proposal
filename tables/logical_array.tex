\begin{center}
\footnotesize
\begin{longtable}{|p{3cm}|p{3.5cm}|p{3.5cm}|p{3.5cm}|}
\caption{Marco lógico de la propuesta} \label{tab:marco_logico} \\ 
\hline
\textbf{Objetivos} & 
\textbf{Indicadores Verificables} & 
\textbf{Medios de Verificación} & 
\textbf{Supuestos} \\ 
\hline
\endfirsthead

\hline
\textbf{Objetivos} & 
\textbf{Indicadores Verificables} & 
\textbf{Medios de Verificación} & 
\textbf{Supuestos} \\ 
\hline
\endhead

\endfoot
\hline
\endlastfoot

% -----------------------------------------------------
% NIVEL 1: FIN (SIN MODIFICACIONES)
% -----------------------------------------------------
\textbf{Fin} \newline 
Contribuir a la comprensión de conceptos fundamentales de interpretación y compilación mediante el uso de un prototipo web. 
&
- Evidencia documental del uso pedagógico. \newline
- Interés del docente y estudiantes en utilizar el prototipo para apoyar la asignatura.
&
- Informe final del trabajo de grado. \newline
- Retroalimentación del docente responsable.
&
- El docente mantiene apertura al uso de recursos digitales. \\ 
\hline

% -----------------------------------------------------
% NIVEL 2: PROPÓSITO (SIN MODIFICACIONES)
% -----------------------------------------------------
\textbf{Propósito} \newline
Disponer de un prototipo web educativo funcional que permita definir gramáticas, visualizar árboles de sintaxis abstracta y representar cambios de ambiente para apoyar el aprendizaje.
&
- Prototipo desplegado y operativo en entorno de pruebas. \newline
- Participación de entre 10 y 20 estudiantes en pruebas de usabilidad. 
&
- Acceso al prototipo. \newline
- Cuestionarios o herramientas similares.
&
- Disposición del docente y estudiantes para participar en la evaluación. \\ 
\hline

% -----------------------------------------------------
% NIVEL 3: COMPONENTES (SIMPLIFICADO)
% -----------------------------------------------------
\multicolumn{4}{|l|}{\textbf{Componentes}} \\ 
\hline

\textbf{Componente 1} \newline
Diagnóstico de dificultades conceptuales.
&
- Documento de diagnóstico elaborado.
&
- Informe y revisión de literatura.
&
- Colaboración del docente. \\ 
\hline

\textbf{Componente 2} \newline
Especificación de requisitos funcionales y no funcionales.
&
- Documento de requisitos validado y priorizado (8-12 F; 5-8 NF).
&
- Documento final de requisitos.
&
- Colaboración del docente en la validación. \\ 
\hline

\textbf{Componente 3} \newline
Diseño de la arquitectura, interfaz y selección tecnológica.
&
- Documento de arquitectura, prototipos de interfaz y justificación técnica.
&
- Diagramas, mockups y documentación técnica.
&
- Acceso a herramientas de diseño. \\ 
\hline

\textbf{Componente 4} \newline
Prototipo web funcional y desplegado.
&
- Prototipo funcional con implementación de requisitos clave. \newline
- Código documentado en repositorio.
&
- Repositorio GitHub y entorno de pruebas.
&
- Capacidad técnica del equipo. \\ 
\hline

\textbf{Componente 5} \newline
Evaluación de funcionalidad y usabilidad.
&
- Cuestionarios aplicados y resultados de observación registrados.
&
- Registros de prueba y análisis cualitativo.
&
- Disposición de estudiantes para participar. \\ 
\hline

% -----------------------------------------------------
% NIVEL 4: ACTIVIDADES (AGRUPADO)
% -----------------------------------------------------
\multicolumn{4}{|l|}{\textbf{Actividades}} \\ 
\hline

\textbf{Actividades C1} \newline
Análisis de material y elaboración del diagnóstico de conceptos difíciles.
&
- Revisión documental e identificación de conceptos.
&
- Informes de revisión, selección de ejemplos y observaciones.
&
- Disponibilidad de información y colaboración del docente. \\ 
\hline

\textbf{Actividades C2} \newline
Identificación, priorización y validación de los requisitos del sistema.
&
- Documento de requisitos finales y criterios de aceptación.
&
- Lista preliminar, documento de criterios y validación docente.
&
- Comprensión de necesidades y retroalimentación del docente. \\ 
\hline

\textbf{Actividades C3} \newline
Diseño de arquitectura, prototipos de interfaz y justificación tecnológica.
&
- Documentos de diseño técnico y prototipos de interfaz creados.
&
- Documentos técnicos (arquitectura, justificación) y prototipos.
&
- Viabilidad técnica y disponibilidad de software de diseño. \\ 
\hline

\textbf{Actividades C4} \newline
Implementación y despliegue del editor, visualizadores (AST y ambiente) e integración.
&
- Prototipo con funcionalidades implementadas y desplegado.
&
- Repositorio con código y pruebas funcionales. \newline
- Demostración funcional del prototipo.
&
- Estabilidad del entorno y servicios de alojamiento. \\ 
\hline

\textbf{Actividades C5} \newline
Diseño y aplicación del protocolo de pruebas de usabilidad y funcionalidad.
&
- Protocolo de pruebas definido y resultados registrados.
&
- Documento de protocolo, cuestionarios y registros de prueba.
&
- Participación voluntaria de estudiantes y docente. \\ \hline

\end{longtable}
\end{center}