\begin{table}[H]
\centering
\setlength{\emergencystretch}{2em}
\begin{tabularx}{\textwidth}{
|>{\raggedright\arraybackslash}p{3cm}
|>{\raggedright\arraybackslash}p{4.2cm}
|>{\raggedright\arraybackslash}X
|>{\raggedright\arraybackslash}X|
}
\hline
\textbf{Autor / Año} & \textbf{Título} & \textbf{Aporte principal} & \textbf{Limitaciones / Vacíos} \\
\hline

Jovanović et al. (2021) &
\textit{ComVIS: A Web-based Tool for Teaching Compiler Construction} &
Herramienta web para visualizar parsing y AST con orientación didáctica. &
No permite edición flexible de gramáticas; cubre solo ciertas fases del compilador; carece de un flujo completo integrado. \\

Lu et al. (2022) &
\textit{Usability of Educational Visualization Tools for Programming Languages} &
Analiza heurísticas de usabilidad educativa para herramientas visuales; aporta criterios de diseño centrado en estudiantes. &
No aborda compiladores directamente; no ofrece propuestas técnicas integradas; alcance conceptual limitado. \\

Del Vado et al. (2023) &
\textit{Interactive Teaching Tutors for Programming Language Foundations} &
Tutores interactivos con visualización y apoyo pedagógico explícito. &
Funcionalidad cerrada; poca flexibilidad técnica; no permite explorar gramáticas ni componentes avanzados. \\

Daleman et al. (2024) &
\textit{ChocoPy: A Pedagogical Python Subset for Compiler Courses} &
Lenguaje educativo con herramientas para mostrar AST, análisis y ejecución. &
No está diseñado como entorno guiado; limitado para principiantes; visualización no centrada en andamiaje pedagógico. \\

Ortin et al. (2022) &
\textit{Teaching Compiler Construction with ANTLR and Modern Tools} &
Buenas prácticas para enseñar compiladores usando herramientas profesionales. &
Curva de aprendizaje alta; herramientas muy complejas para novatos; escasa interactividad visual pedagógica. \\

\hline
\end{tabularx}
\vspace{2pt}
\caption{Síntesis comparativa de las principales fuentes utilizadas, incluyendo sus aportes y limitaciones en relación con la enseñanza de compiladores. }
\label{tab:fuentes}
\end{table}
