El presente anteproyecto propone el desarrollo de un prototipo web interactivo con el objetivo de apoyar la comprensión de los contenidos de la asignatura \textit{Fundamentos de Interpretación y Compilación de Lenguajes de Programación} (LP) en la Universidad del Valle. Esta iniciativa surge como respuesta a las dificultades que experimentan los estudiantes para asimilar conceptos que presentan un alto nivel de abstracción, a causa de la ausencia de herramientas didácticas interactivas que permitan la visualización práctica y dinámica de los procesos internos.

El prototipo se enfocará en el uso de lenguajes orientados por sintaxis, permitiendo a los usuarios definir y editar gramáticas, visualizar los árboles de sintaxis abstracta (AST) generados y representar los cambios en el ambiente durante la evaluación de programas. Asimismo, se enmarca en principios de usabilidad y enfoques constructivistas del aprendizaje, promoviendo la interacción activa del estudiante con los contenidos y fomentando la comprensión mediante ejemplos visuales y ejercicios prácticos.
