Teniendo en cuenta que el uso responsable de la energía es un componente fundamental dentro de las acciones de sostenibilidad promovidas en el país, y que ANDESCO resalta la importancia de adoptar buenas prácticas ambientales en todos los sectores \cite{andesco2016}, este proyecto considera la estimación del consumo energético derivado del uso de los dispositivos necesarios para su desarrollo, con el fin de evaluar su impacto ambiental.

\begin{center}
\footnotesize
\begin{longtable}{|p{5cm}|c|c|c|c|}
\caption[Dispositivos previstos para la operación]{Dispositivos previstos para la operación} \\
\hline
\multicolumn{5}{|c|}{\textbf{Características técnicas de los dispositivos}} \\ \hline
\textbf{Dispositivos consumidores} & \textbf{Cantidad} & \textbf{Watts (activo)} & \textbf{Watts (suspensión)} & \textbf{Watts (apagado)} \\ \hline
\endfirsthead

\hline
\textbf{Dispositivos consumidores} & \textbf{Cantidad} & \textbf{Watts (activo)} & \textbf{Watts (suspensión)} & \textbf{Watts (apagado)} \\ \hline
\endhead

MacBook Air M1 & 1 & 30 & 1 & 0.3 \\ \hline
HP Laptop 14-cm0xxx & 1 & 45 & 2 & 0.5 \\ \hline
Conexión a internet (router) & 2 & 15 & 0 & 0 \\ \hline
Luces LED & 2 & 20 & 0 & 0 \\ \hline
\textbf{Total} & 6 & 110 W & 3 W & 0.8 W \\ \hline

\end{longtable}
\end{center}


Por otra parte, se proyecta que durante el desarrollo del proyecto cada estudiante utilizará su computador portátil aproximadamente 6 horas al día, mientras que la conexión a internet permanecerá activa las 24 horas. Adicionalmente, cada estudiante emplea iluminación LED durante aproximadamente 6 horas al día en su espacio de trabajo. Estas estimaciones permiten calcular de manera precisa el consumo energético asociado a las fases iniciales de ejecución.

\begin{table}[H]
\centering
\begin{tabular}{|p{3.2cm}|c|c|c|c|c|c|c|c|}
\hline
\textbf{Dispositivo} & \textbf{(1)} & \textbf{(2)} & \textbf{(3)} & \textbf{(4)} &
\textbf{(5)} & \textbf{(6)} & \textbf{(7)} & \textbf{(8)} \\ \hline

MacBook Air M1 & 6  & 0.180 & 1  & 0.0010 & 17 & 0.005 & 0.180 & 0.0061 \\ \hline
HP Laptop 14-cm0xxx & 6  & 0.258 & 1  & 0.0020 & 17 & 0.009 & 0.258 & 0.0105 \\ \hline
Router & 24 & 0.360 & 0  & 0.0000 & 0  & 0.000 & 0.360 & 0.0000 \\ \hline
Luces LED & 6  & 0.120 & 18 & 0.0000 & 18 & 0.000 & 0.120 & 0.0000 \\ \hline

\multicolumn{7}{|c|}{\textbf{Total}} & \textbf{0.918} & \textbf{0.0166} \\ \hline

\multicolumn{8}{|c|}{\textbf{Total día}} & \textbf{0.9346 kWh} \\ \hline
\end{tabular}
\vspace{4pt}
\caption{Consumo energético diario estimado de los dispositivos.}
\end{table}


En la Tabla anterior, las columnas (1) y (2) corresponden al número de horas diarias en que cada dispositivo opera bajo condiciones normales de funcionamiento y al consumo energético asociado en kilovatios-hora (kWh). Las columnas (3) y (4) representan el tiempo y el consumo registrado mientras el dispositivo permanece en estado de suspensión o hibernación. De manera similar, las columnas (5) y (6) indican las horas y el consumo cuando el equipo está apagado. Finalmente, las columnas (7) y (8) muestran el consumo real total diario y el consumo fantasma estimado, respectivamente.

Con el supuesto de un trabajo de cinco días a la semana con dos días de descanso se presentan los consumos semanales, fantasma, mensual y anual.

\begin{table}[H]
\centering
\begin{tabular}{|c|c|c|c|}
\hline
\textbf{Consumo Semanal (kWh)} & 
\textbf{Consumo Fantasma (kWh)} & 
\textbf{Consumo Mensual (kWh)} & 
\textbf{Consumo Anual (kWh)} \\ \hline

4.673 & 0.0332 & 18.8248 & 150.864 \\ \hline

\end{tabular}
\vspace{4pt}
\caption{Consumo energético totalizados}
\end{table}


En términos de emisiones y huella de carbono, el cálculo se realizó aplicando la metodología propuesta en el Protocolo Global Para Inventarios de Emisión de Gases de Efecto Invernadero a Escala Comunitaria \cite{keanfong2012} y empleando el factor de emisión oficial del Sistema Interconectado Nacional reportado por la UPME \cite{upme2019}. A partir de estas fuentes, se estimó la emisión equivalente de CO$_2$ asociada al consumo energético anual del proyecto.

\begin{table}[H]
\centering
\begin{tabular}{|c|c|c|c|}
\hline
\textbf{Gas de efecto invernadero considerado} & \textbf{Factor de Emisión} & 
\textbf{Consumo anual} & \textbf{Tn CO$_2$e / GWh} \\ \hline

CO$_2$ & 0.13 & 0.150864 & 0.01961 \\ \hline

\end{tabular}
\vspace{2pt}
\caption{Impacto del consumo energético en gases de efecto invernadero}
\end{table}


Con base en el consumo energético anual estimado, el proyecto genera 0,01961 toneladas de CO$_2$e al año, una cifra muy baja según la literatura, por lo que no se requiere un proceso formal de compensación. No obstante, es recomendable implementar acciones para reducir el impacto ambiental, tales como apagar y desconectar los computadores y las luces LED al finalizar las jornadas de trabajo. Además, aunque el router requiere funcionamiento continuo, su gestión eficiente también contribuye a minimizar el consumo. Estas medidas permiten disminuir el gasto energético diario y, en consecuencia, la huella de carbono asociada al proyecto.
